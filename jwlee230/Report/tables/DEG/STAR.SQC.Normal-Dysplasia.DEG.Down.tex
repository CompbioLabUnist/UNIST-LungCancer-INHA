\begin{tabular}{lrrr}
\toprule
  gene &  log2FoldChange &   pvalue &     padj \\
\midrule
TDRD12 &       -6.17e+00 & 1.73e-09 & 2.38e-06 \\
AKR1C2 &       -5.85e+00 & 1.61e-10 & 3.34e-07 \\
 SYT14 &       -5.78e+00 & 8.90e-11 & 2.21e-07 \\
\bottomrule
\end{tabular}
