% !TeX spellcheck = en_US
% !TeX encoding = UTF-8
\documentclass{beamer}

\mode<presentation> { \usetheme{Madrid} }

\usepackage{graphicx, graphics}
\usepackage[notocbib]{apacite}
\usepackage[style=iso]{datetime2}
\usepackage{enumerate}
\DeclareGraphicsExtensions{.pdf, .png, .jpg, .gif}

\AtBeginSection[]
{
    \begin{frame}
        \vfill
        \centering
        \begin{beamercolorbox}[sep=8pt, center, shadow=true, rounded=true]{title}
            \usebeamerfont{title}
            \insertsectionhead
            \par
        \end{beamercolorbox}
        \vfill
    \end{frame}
}

\AtBeginSubsection[]
{
    \begin{frame}
        \vfill
        \centering
        \begin{beamercolorbox}[sep=8pt, center, shadow=true, rounded=true]{title}
            \usebeamerfont{title}
            \insertsectionhead
            \par
        \end{beamercolorbox}
        \begin{beamercolorbox}[sep=4pt, center, shadow=true, rounded=true]{title}
            \usebeamerfont{subtitle}
            \insertsubsectionhead
            \par
        \end{beamercolorbox}
        \vfill
    \end{frame}
}

\title[Lung Precancer]{Lung Precancer Study}

\author[Jaewoong Lee]
{
    Jaewoong Lee
    \and
    Y. Choi
    \and
    I. Yun
    \and
    S. Park
    \and
    Semin Lee
}

\institute[UNIST BME]
{
    Department of Biomedical Engineering
    \newline
    Ulsan National Institute of Science and Technology
    \medskip
    \newline
    \textit{jwlee230@unist.ac.kr}
}

\date{\today}

\begin{document}
    \begin{frame}
        \titlepage
    \end{frame}

    \begin{frame}
        \frametitle{Overview}
        \tableofcontents[hideallsubsections]
    \end{frame}

    \section{Introduction}
    \subsection{Lung Cancer}
    \begin{frame}[allowframebreaks]
        \frametitle{Lung Cancer?}

        \begin{block}{The most common cancer}
            The most common form of cancer: \\
            12.3 \% of all cancers \cite{lung3}
        \end{block}

        \begin{block}{The most important factor}
            \textbf{Tobacco}
        \end{block}
    \end{frame}

    \begin{frame}
        \frametitle{Cancer Survival Rate in Korea}

        \begin{figure}
            \includegraphics[width=\linewidth]{figures/LungCancer/rate.png}
            \caption{Common cancer survival rates \protect\cite{lung6}}
        \end{figure}

        \begin{block}{Survival rate (More than 5 yr)}
            \begin{itemize}
                \item Tyroid: 68.4 \%
                \item Lung: 35.4 \%
            \end{itemize}
        \end{block}
    \end{frame}

    \begin{frame}
        \frametitle{Type of Lung Cancer}

        Types of lung cancer:
        \begin{enumerate}
            \item Adenocarcinoma (ADC) (40 \%) $\bigstar$
            \item Squamous cell carcinoma (SQC) (25 \%) $\bigstar$
            \item Small cell carcinoma (20 \%)
            \item Large cell carcinoma (10 \%)
            \item Adenosquamous carcinoma ($<$ 5 \%)
            \item Carcinoid ($<$ 5 \%)
            \item Bronchioalveolar (Bronchial gland carcinoma)
        \end{enumerate}
        \cite{lung1, lung2}
    \end{frame}

    \begin{frame}[allowframebreaks]
        \frametitle{ADC vs. SQC}

        \begin{figure}
            $\begin{array}{cc}
                \includegraphics[width=0.3 \linewidth]{figures/LungCancer/ADC.png}
                &
                \includegraphics[width=0.3 \linewidth]{figures/LungCancer/SQC.png}
                \\
                \mbox{(a) ADC} & \mbox{(b) SQC} \\
            \end{array}$
            \caption{ADC and SQC histology in Lung cancer \cite{lung4}}
        \end{figure}

        \begin{figure}
            $\begin{array}{cc}
                \includegraphics[width=0.4 \linewidth]{figures/LungCancer/wang1.png}
                &
                \includegraphics[width=0.4 \linewidth]{figures/LungCancer/wang2.png}
                \\
                \mbox{(a) All patients} & \mbox{(b) By cancer stages} \\
            \end{array}$
            \caption{Kaplan-Meiere survival curves for ADC \& SQC \cite{lung5}}
        \end{figure}

        \begin{block}{Findings}
            SQC is more dangerous than ADC. $\because p < 0.001$
        \end{block}
    \end{frame}

    \subsection{Study Objectives}
    \begin{frame}
        \frametitle{Study Objectives}

        \begin{block}{Find different mutations}
            \begin{itemize}
                \item between WES vs. WTS
                \item from cancer vs. precancer
            \end{itemize}
        \end{block}

        \begin{block}{Pathway examine}
            \begin{itemize}
                \item with the mutation of WES \& RNA-seq
                \item with immune-depleted animal models
            \end{itemize}
        \end{block}

        \begin{block}{Ultra-deep sequencing}
            to find an \textit{infinitesimal} quantity of Non-Circulating Tumor DNA
            \begin{itemize}
                \item from blood
                \item from urine
                \item from bronchus
            \end{itemize}
        \end{block}
    \end{frame}

    \section{Materials}
    \begin{frame}
        \frametitle{Lung Cancer Data}

        \begin{itemize}
            \item WES (n=289) + Transcriptome (n=166)
            \item Normal + \{Primary, CIS + AIS, AAH, Dysplasia, MIA\}
            \begin{itemize}
                \item Carcinoma in situ
                \item Adenocarcinoma in situ
                \item Atypical adenomatous hyperplasia
                \item Dysplasia
                \item Minimally invasive adenocarcinoma
            \end{itemize}
            \item Squamouse cell carcinoma (SQC) \& Adenocarcinoma (ADC)
            \begin{enumerate}
                \item Normal $\rightarrow$ Dysplasia $\rightarrow$ CIS $\rightarrow$ SQC (n=80)
                \item Normal $\rightarrow$ AAH $\rightarrow$ AIS $\rightarrow$ MIA $\rightarrow$ ADC (n=28)
            \end{enumerate}
        \end{itemize}
    \end{frame}

    \section{Methods}
    \subsection{Workflows}
    \begin{frame}
        \frametitle{Data pre-processing for variant discovery}

        \begin{figure}
            \includegraphics[width=0.3 \linewidth]{figures/Workflow/mapping.png}
            \caption{Data pre-processing for variant discovery \protect\cite{gatk1, gatk2}}
        \end{figure}
    \end{frame}

    \begin{frame}
        \frametitle{Somatic short variant discovery}

        \begin{figure}
            \includegraphics[width=0.7 \linewidth]{figures/Workflow/somatic_short_variants.png}
            \caption{Somatic short variant (SNVs + Indels) discovery workflow \protect\cite{gatk1, gatk2}}
        \end{figure}
    \end{frame}

    \begin{frame}
        \frametitle{Germline short variant discovery}

        \begin{figure}
            \includegraphics[width=0.7 \linewidth]{figures/Workflow/germline_short_variant.png}
            \caption{Germline short variant (SNVs + Indels) discovery workflow \protect\cite{gatk1, gatk2}}
        \end{figure}
    \end{frame}

    \begin{frame}
        \frametitle{RNA-seq short variant discovery}

        \begin{figure}
            \includegraphics[width=0.8 \linewidth]{figures/Workflow/RNA_short_variant.png}
            \caption{RNA-seq short variant (SNVs + Indels) discovery workflow \protect\cite{gatk1, gatk2}}
        \end{figure}
    \end{frame}

    \section{Results}
    \subsection{Quality Checks}
    \begin{frame}
        \frametitle{FastQC?}

        \begin{figure}
            \includegraphics[width=0.5 \linewidth]{figures/FastQC/example.png}
            \caption{Example of FastQC Result \protect\cite{fastqc1}}
        \end{figure}

        \begin{itemize}
            \item A quality check tool for sequence data
            \item Give an overview that which test may be problems
        \end{itemize}
    \end{frame}

    \begin{frame}
        \frametitle{FastQC on WES}

        \begin{figure}
            \includegraphics[width=\linewidth]{figures/FastQC/FastQC_WES.pdf}
            \caption{FastQC with WES data}
        \end{figure}

        \begin{alertblock}{Failure on 33P1 sample}
            33P1 is excluded at further analysis.
        \end{alertblock}
    \end{frame}

    \begin{frame}[allowframebreaks]
        \frametitle{Failure on 33P1}

        \begin{figure}
            $\begin{array}{cc}
                \includegraphics[width=0.4 \linewidth]{figures/FastQC/33N.png}
                &
                \includegraphics[width=0.4 \linewidth]{figures/FastQC/33P1.png}
                \\
                \mbox{(a) 33N} & \mbox{(b) 33P1} \\
            \end{array}$
            \caption{Per Base Sequence Quality Results}
        \end{figure}

        \begin{figure}
            \includegraphics[width=\linewidth]{figures/FastQC/BWA.png}
            \caption{Coverage Depth Plot}
        \end{figure}
    \end{frame}

    \begin{frame}
        \frametitle{FastQC on WTS}

        \begin{figure}
            \includegraphics[width=\linewidth]{figures/FastQC/FastQC_WTS.pdf}
            \caption{FastQC with WTS data}
        \end{figure}

        \begin{exampleblock}{All sample are good to analysis}
            $\because$ No sample has more than 5 failures.
        \end{exampleblock}
    \end{frame}

    \subsection{Copy Number Variations}
    \begin{frame}
        \frametitle{Sequenza?}

        \begin{figure}
            \includegraphics[width=0.6 \linewidth]{figures/Workflow/sequenza.jpg}
            \caption{Representative Output of the Sequenza \protect\cite{sequenza1}}
        \end{figure}
    \end{frame}

    \begin{frame}
        \frametitle{Cellularity \& Ploidy on WES}

        \begin{figure}
            $\begin{array}{cc}
                \includegraphics[width=0.4 \linewidth]{figures/Sequenza/BWA-sequenza-SQC.pdf}
                &
                \includegraphics[width=0.4 \linewidth]{figures/Sequenza/BWA-sequenza-ADC.pdf}
                \\
                \mbox{(a) SQC Samples} & \mbox{(b) ADC Samples} \\
            \end{array}$
            \caption{Cellularity and Ploidy from Sequenza}
        \end{figure}
    \end{frame}

    \begin{frame}
        \frametitle{Genome View on Patient \#57}

        \begin{figure}
            \includegraphics[width=\linewidth]{figures/Sequenza/57D1.pdf}
            \includegraphics[width=\linewidth]{figures/Sequenza/57C1.pdf}
            \includegraphics[width=\linewidth]{figures/Sequenza/57P1.pdf}
            \caption{Dysplasia-CIS-Primary Tumor on Patient \#57}
        \end{figure}
    \end{frame}

    \begin{frame}
        \frametitle{CNVs of SQC}

        \begin{figure}
            \includegraphics[height=0.7 \textheight]{figures/Sequenza/BWA-genome-SQC.pdf}
            \caption{CNV Plot with SQC Patients}
        \end{figure}
    \end{frame}

    \begin{frame}
        \frametitle{CNVs of ADC}

        \begin{figure}
            \includegraphics[height=0.7 \textheight]{figures/Sequenza/BWA-genome-ADC.pdf}
            \caption{CNV Plot with ADC Patients}
        \end{figure}
    \end{frame}

    \begin{frame}
        \frametitle{SQC vs. ADC}

        \begin{figure}
            \includegraphics[width=\linewidth]{figures/Sequenza/BWA-simple-SQC.pdf}
            \caption{Simple CNV Plot with SQC Patients}
        \end{figure}

        \begin{figure}
            \includegraphics[width=\linewidth]{figures/Sequenza/BWA-simple-ADC.pdf}
            \caption{Simple CNV Plot with ADC Patients}
        \end{figure}
    \end{frame}

    \begin{frame}
        \frametitle{Findings in Sequenza}

        \begin{exampleblock}{Sequenza Findings}
            \begin{itemize}
                \item SQC have more CNVs than ADC.
                \item SQC have aggressive CNVs on chromosome \#3 and \#5.
                \item ADC have aggressive CNVs on chromosome \#5.
            \end{itemize}
        \end{exampleblock}
    \end{frame}

    \subsection{SNVs Analysis}
    \begin{frame}
        \frametitle{Mutect2?}

        \begin{figure}
            \includegraphics[width=0.6 \linewidth]{figures/Workflow/somatic_short_variants.png}
            \caption{Somatic short variant discovery workflow \protect\cite{gatk1, gatk2}}
        \end{figure}
    \end{frame}

    \begin{frame}
        \frametitle{MutEnricher?}

        \begin{figure}
            \includegraphics[width=0.8 \linewidth]{figures/Workflow/MutEnricher.jpg}
            \caption{Schematic representation of MunEnricher's analysis procedures \protect\cite{MutEnricher1}}
        \end{figure}
    \end{frame}

    \begin{frame}
        \frametitle{Somatic Variant in SQC}

        \begin{figure}
            \includegraphics[width=\linewidth]{figures/Mutect2/BWA-SQC.pdf}
            \caption{CoMut Plot with SQC Patients}
        \end{figure}
    \end{frame}

    \begin{frame}
        \frametitle{Somatic Variant in ADC}

        \begin{figure}
            \includegraphics[width=0.5 \linewidth]{figures/Mutect2/BWA-ADC.pdf}
            \caption{CoMut Plot with ADC Patients}
        \end{figure}
    \end{frame}

    \begin{frame}
        \frametitle{Findings in SNVs Analysis}

        \begin{exampleblock}{TTN}
            TTN is the most mutated gene both in SQC and ADC.
        \end{exampleblock}
    \end{frame}

    \subsection{VAF Analysis}
    \begin{frame}
        \frametitle{VAF?}

        \begin{itemize}
            \item Variant allele frequency
            \item $\textrm{VAF} = \textrm{Alternative allele read count} / \textrm{Total read count}$
            \item To find tumor evolution
        \end{itemize}

        \begin{figure}
            \includegraphics[width=0.4 \linewidth]{figures/VAF/VAF.jpg}
            \caption{VAF distribution of validated mutations \protect\cite{VAF1}}
        \end{figure}
    \end{frame}

    \begin{frame}[allowframebreaks]
        \frametitle{VAF Plots}
    \end{frame}

    \begin{frame}
        \frametitle{PyClone?}

        \begin{figure}
            \includegraphics[width=0.8 \linewidth]{figures/Workflow/PyClone.jpg}
            \caption{Analysis of multiple samples by PyClone \protect\cite{pyclone1}}
        \end{figure}
    \end{frame}

    \begin{frame}[allowframebreaks]
        \frametitle{PyClone Plots}
    \end{frame}

    \begin{frame}
        \frametitle{Findings in VAF Analysis}
    \end{frame}

    \subsection{Differences in Gene Expression Levels}
    \begin{frame}
        \frametitle{RSEM?}

        \begin{figure}
            \includegraphics[width=0.2 \linewidth]{figures/Workflow/RSEM.jpg}
            \caption{RSEM workflow \protect\cite{RSEM1}}
        \end{figure}
    \end{frame}

    \begin{frame}
        \frametitle{DESeq2?}

        \begin{figure}
            \includegraphics[width=0.5 \linewidth]{figures/Workflow/DESeq2.png}
            \caption{DESeq2 workflow \protect\cite{DESeq1}}
        \end{figure}
    \end{frame}

    \begin{frame}
        \frametitle{Enrichr?}

        \begin{figure}
            \includegraphics[width=0.8 \linewidth]{figures/Workflow/Enrichr.jpg}
            \caption{Enrichr workflow \protect\cite{Enrichr1}}
        \end{figure}
    \end{frame}

    \begin{frame}
        \frametitle{DEG Selection Strategy}
        DEG: differentially expressed genes

        \begin{block}{Fold Change}
            $\log_{2}(\text{Fold Change}) > 1 \vee \log_{2}(\text{Fold Change}) < -1$
        \end{block}

        \begin{block}{P-value}
            $\text{P-value} < 0.05$
        \end{block}

        \begin{block}{Adjusted P-value}
            $\text{Padj} < 0.05$
        \end{block}
    \end{frame}

    \begin{frame}
        \frametitle{DEG Volcano Plots in SQC}

        Normal $\rightarrow$ Dysplasia $\rightarrow$ CIS $\rightarrow$ Primary (SQC)

        \begin{figure}
            $\begin{array}{ccc}
                \includegraphics[width=0.2 \linewidth]{figures/DEG/Volcano/SQC.STAR.Normal-Dysplasia.volcano.pdf}
                &
                \includegraphics[width=0.2 \linewidth]{figures/DEG/Volcano/SQC.STAR.Normal-CIS.volcano.pdf}
                &
                \includegraphics[width=0.2 \linewidth]{figures/DEG/Volcano/SQC.STAR.Normal-Primary.volcano.pdf}
                \\
                \mbox{(a) Normal-Dysplasia} & \mbox{(b) Normal-CIS} & \mbox{(c) Normal-Primary} \\

                \includegraphics[width=0.2 \linewidth]{figures/DEG/Volcano/SQC.STAR.Dysplasia-CIS.volcano.pdf}
                &
                \includegraphics[width=0.2 \linewidth]{figures/DEG/Volcano/SQC.STAR.Dysplasia-Primary.volcano.pdf}
                &
                \includegraphics[width=0.2 \linewidth]{figures/DEG/Volcano/SQC.STAR.CIS-Primary.volcano.pdf}
                \\
                \mbox{(d) Dysplasia-CIS} & \mbox{(e) Dysplasia-Primary} & \mbox{(f) CIS-Primary} \\
            \end{array}$
            \caption{DEG Volcano Plots in SQC}
        \end{figure}
    \end{frame}

    \begin{frame}
        \frametitle{DEG Venn Diagram with Bowtie2 in SQC}

        Normal $\rightarrow$ Dysplasia $\rightarrow$ CIS $\rightarrow$ Primary (SQC)

        \begin{figure}
            $\begin{array}{cc}
                \includegraphics[width=0.4 \linewidth]{figures/DEG/Pair-Venn/SQC.Up.STAR.pair-venn.pdf}
                &
                \includegraphics[width=0.4 \linewidth]{figures/DEG/Pair-Venn/SQC.Down.STAR.pair-venn.pdf}
                \\
                \mbox{(a) Up-regulated} & \mbox{(b) Down-regulated} \\
            \end{array}$
            \caption{DEG Venn Diagram in SQC}
        \end{figure}
    \end{frame}

    \begin{frame}
        \frametitle{Enrichment test with Enrichr in SQC}
    \end{frame}

    \begin{frame}
        \frametitle{DEG Volcano Plots with Bowtie2 in ADC}
        Normal $\rightarrow$ AAH $\rightarrow$ AIS $\rightarrow$ MIA $\rightarrow$ Primary (ADC)

        \begin{figure}
            $\begin{array}{ccc}
                \includegraphics[width=0.2 \linewidth]{figures/DEG/Volcano/ADC.STAR.Normal-AAH.volcano.pdf}
                &
                \includegraphics[width=0.2 \linewidth]{figures/DEG/Volcano/ADC.STAR.Normal-AIS.volcano.pdf}
                &
                \includegraphics[width=0.2 \linewidth]{figures/DEG/Volcano/ADC.STAR.Normal-Primary.volcano.pdf}
                \\
                \mbox{(a) Normal-AAH} & \mbox{(b) Normal-AIS} & \mbox{(c) Normal-Primary} \\

                \includegraphics[width=0.2 \linewidth]{figures/DEG/Volcano/ADC.STAR.AAH-AIS.volcano.pdf}
                &
                \includegraphics[width=0.2 \linewidth]{figures/DEG/Volcano/ADC.STAR.AAH-Primary.volcano.pdf}
                &
                \includegraphics[width=0.2 \linewidth]{figures/DEG/Volcano/ADC.STAR.AIS-Primary.volcano.pdf}
                \\
                \mbox{(d) AAH-AIS} & \mbox{(e) AAH-Primary} & \mbox{(f) AIS-Primary} \\
            \end{array}$
            \caption{DEG Volcano Plots in ADC}
        \end{figure}
    \end{frame}

    \begin{frame}
        \frametitle{DEG Venn Diagram with Bowtie2 in ADC}

        Normal $\rightarrow$ AAH $\rightarrow$ AIS $\rightarrow$ MIA $\rightarrow$ Primary (ADC)

        \begin{figure}
            $\begin{array}{cc}
                \includegraphics[width=0.4 \linewidth]{figures/DEG/Pair-Venn/ADC.Up.STAR.pair-venn.pdf}
                &
                \includegraphics[width=0.4 \linewidth]{figures/DEG/Pair-Venn/ADC.Down.STAR.pair-venn.pdf}
                \\
                \mbox{(a) Up-regulated} & \mbox{(b) Down-regulated} \\
            \end{array}$
            \caption{DEG Venn Diagram in ADC}
        \end{figure}
    \end{frame}

    \begin{frame}
        \frametitle{Enrichment test with Enrichr in ADC}
    \end{frame}

    \begin{frame}
        \frametitle{Findings in DEG Analysis}
    \end{frame}

    \subsection{Bulk Cell Deconvolution}
    \begin{frame}
        \frametitle{Single-cell data as Reference}

        \begin{figure}
            \includegraphics[width=0.8 \linewidth]{figures/CIBERSORTx/reference.jpg}
            \caption{Comprehensive dissection and clustering of 208,506 single cells from LUAD patients \protect\cite{singlecell1}}
        \end{figure}
    \end{frame}

    \begin{frame}
        \frametitle{CIBERSORTx}

        \begin{figure}
            \includegraphics[width=0.6 \linewidth]{figures/Workflow/CIBERSORTx.png}
            \caption{Workflow for CIBERSORTx \protect\cite{cibersort1, cibersort2}}
        \end{figure}
    \end{frame}

    \begin{frame}
        \frametitle{Cluster Plot with Bowtie2 in ADC}

        \begin{figure}
            \includegraphics[height=0.7 \textheight]{figures/CIBERSORTx/clustermap/ADC.Bowtie2.cluster.pdf}
            \caption{Cluster Plot in ADC}
        \end{figure}
    \end{frame}

    \begin{frame}
        \frametitle{Cluster Plot with Bowtie2 in SQC}

        \begin{figure}
            \includegraphics[width=0.8 \linewidth]{figures/CIBERSORTx/clustermap/SQC.Bowtie2.cluster.pdf}
            \caption{Cluster Plot in SQC}
        \end{figure}
    \end{frame}

    \begin{frame}
        \frametitle{Benchmarking of Cell Deconvolution Tools}

        \begin{figure}
            \includegraphics[width=0.6 \linewidth]{figures/Workflow/Deconvolution.png}
            \caption{Memory and time requirements for the cell deconvolution methods \protect\cite{deconvolution1}}
        \end{figure}

        \begin{block}{Top 3 Methods}
            \begin{enumerate}
                \item BisqueRNA \cite{Bisque1}
                \item MuSiC \cite{MuSiC1}
                \item SCDC \cite{SCDC1}
            \end{enumerate}
        \end{block}
    \end{frame}

    \begin{frame}
        \frametitle{BisqueRNA?}

        \begin{figure}
            \includegraphics[width=0.6 \linewidth]{figures/Workflow/Bisque.jpg}
            \caption{Graphical overview of the Bisque decomposition methods \protect\cite{Bisque1}}
        \end{figure}
    \end{frame}

    \begin{frame}
        \frametitle{MuSiC?}

        \begin{figure}
            \includegraphics[width=\linewidth]{figures/Workflow/MuSiC.jpg}
            \caption{Overview of MuSiC framework \protect\cite{MuSiC1}}
        \end{figure}
    \end{frame}

    \begin{frame}
        \frametitle{SCDC?}

        \begin{figure}
            \includegraphics[width=\linewidth]{figures/Workflow/SCDC.png}
            \caption{Overview of deconvolution by SCDC \protect\cite{SCDC1}}
        \end{figure}
    \end{frame}

    \begin{frame}
        \frametitle{Findings in Bulk Cell Deconvolution}
    \end{frame}

    \subsection{Tumor Evolution Trajectories Analysis}
    \begin{frame}
        \frametitle{Revolver?}

        \begin{figure}
            \includegraphics[width=0.8 \linewidth]{figures/Workflow/revolver.jpg}
            \caption{Repeated Evolutionary Trajectories \cite{revolver1}}
        \end{figure}
    \end{frame}

    \begin{frame}
        \frametitle{Findings in Tumor Evolution Trajectories Analysis}
    \end{frame}

    \subsection{Discovery of Gene Fusion}
    \begin{frame}
        \frametitle{Arriba?}

        \begin{figure}
            \includegraphics[width=0.6 \linewidth]{figures/Workflow/Arriba.png}
            \caption{Benchmark of Arriba versus alternative methods \protect\cite{Arriba1}}
        \end{figure}
    \end{frame}

    \begin{frame}
        \frametitle{FusionCatcher?}

        \begin{figure}
            \includegraphics[width=0.6 \linewidth]{figures/Workflow/FusionCatcher.png}
            \caption{FusionCatcher \protect\cite{FusionCatcher1}}
        \end{figure}
    \end{frame}

    \begin{frame}
        \frametitle{STAR-fusion?}

        \begin{figure}
            \includegraphics[width=0.4 \linewidth]{figures/Workflow/STARfusion.jpg}
            \caption{Methods for fusion transcript prediction and accuracy evaluation \protect\cite{STARfusion1}}
        \end{figure}
    \end{frame}

    \begin{frame}
        \frametitle{Findings in Gene Fusion Discovery}
    \end{frame}

    \section{Discussion}

    \section{References}
    \begin{frame}[allowframebreaks]
        \frametitle{References}
        \bibliographystyle{apacite}
        \bibliography{reference}
    \end{frame}
\end{document}