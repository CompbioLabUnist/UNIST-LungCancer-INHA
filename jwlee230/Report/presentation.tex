% !TeX spellcheck = en_US
% !TeX encoding = UTF-8
\documentclass{beamer}

\mode<presentation> { \usetheme{Madrid} }

\usepackage{graphicx, graphics}
\usepackage[notocbib]{apacite}
\usepackage[style=iso]{datetime2}
\usepackage{enumerate}
\DeclareGraphicsExtensions{.pdf, .png, .jpg, .gif}

\AtBeginSection[]
{
    \begin{frame}
        \vfill
        \centering
        \begin{beamercolorbox}[sep=8pt, center, shadow=true, rounded=true]{title}
            \usebeamerfont{title}
            \insertsectionhead
            \par
        \end{beamercolorbox}
        \vfill
    \end{frame}
}

\AtBeginSubsection[]
{
    \begin{frame}
        \vfill
        \centering
        \begin{beamercolorbox}[sep=8pt, center, shadow=true, rounded=true]{title}
            \usebeamerfont{title}
            \insertsectionhead
            \par
        \end{beamercolorbox}
        \begin{beamercolorbox}[sep=4pt, center, shadow=true, rounded=true]{title}
            \usebeamerfont{subtitle}
            \insertsubsectionhead
            \par
        \end{beamercolorbox}
        \vfill
    \end{frame}
}

\title[Lung Precancer]{Lung Precancer Study}

\author[Jaewoong Lee]
{
    Jaewoong Lee
    \and
    Y. Choi
    \and
    I. Yun
    \and
    S. Park
    \and
    Semin Lee
}

\institute[UNIST BME]
{
    Department of Biomedical Engineering
    \newline
    Ulsan National Institute of Science and Technology
    \medskip
    \newline
    \textit{jwlee230@unist.ac.kr}
}

\date{\today}

\begin{document}
    \begin{frame}
        \titlepage
    \end{frame}

    \begin{frame}
        \frametitle{Overview}
        \tableofcontents[hideallsubsections]
    \end{frame}

    \section{Introduction}
    \subsection{Lung Cancer}
    \begin{frame}
        \frametitle{Lung Cancer?}

        \begin{itemize}
            \item The most common form of cancer (12.3 \% of all cancers) \cite{lung3}
            \item The most important factor: \textbf{Tobacco}
        \end{itemize}
    \end{frame}

    \begin{frame}
        \frametitle{Type of Lung Cancer}

        Types of lung cancer:
        \begin{itemize}
            \item Adenocarcinoma (ADC) (40 \%) $\bigstar$
            \item Squamous cell carcinoma (SQC) (25 \%) $\bigstar$
            \item Small cell carcinoma (20 \%)
            \item Large cell carcinoma (10 \%)
            \item Adenosquamous carcinoma
            \item Carcinoid
            \item Bronchioalveolar (Bronchial gland carcinoma)
        \end{itemize}
        \cite{lung1, lung2}
    \end{frame}

    \begin{frame}[allowframebreaks]
        \frametitle{ADC vs. SQC}

        \begin{figure}
            $\begin{array}{cc}
                \includegraphics[width=0.3 \linewidth]{figures/LungCancer/ADC.png}
                &
                \includegraphics[width=0.3 \linewidth]{figures/LungCancer/SQC.png}
                \\
                \mbox{(a) ADC} & \mbox{(b) SQC} \\
            \end{array}$
            \caption{ADC and SQC histology in Lung cancer \cite{lung4}}
        \end{figure}

        \begin{figure}
            $\begin{array}{cc}
                \includegraphics[width=0.4 \linewidth]{figures/LungCancer/wang1.png}
                &
                \includegraphics[width=0.4 \linewidth]{figures/LungCancer/wang2.png}
                \\
                \mbox{(a) All patients} & \mbox{(b) By cancer stages} \\
            \end{array}$
            \caption{Kaplan-Meiere survival curves for ADC \& SQC \cite{lung5}}
        \end{figure}
        $\therefore$ In every plots, $p < 0.001$ \\
        $\therefore$ SQC is more dangerous than ADC.
    \end{frame}

    \subsection{Study Objectives}
    \begin{frame}
        \frametitle{Study Objectives}

        \begin{itemize}
            \item Find different mutations
            \begin{itemize}
                \item between WES
                \item between WTS
            \end{itemize}
            \begin{itemize}
                \item from cancer
                \item from precancer
            \end{itemize}

            \item Pathway examine from the mutations
            \begin{itemize}
                \item of WES
                \item of RNA-seq
            \end{itemize}

            \item Ultra-deep sequencing to find an \textit{infinitesimal} quantity of Non-Circulating Tumor DNA
            \begin{itemize}
                \item from blood
                \item from urine
                \item frrom bronchus
            \end{itemize}

            \item Diagnostic performace
        \end{itemize}
    \end{frame}

    \section{Materials}
    \begin{frame}
        \frametitle{Lung Cancer Data}

        \begin{itemize}
            \item WES (n=289) + Transcriptome (n=166)
            \item Normal + \{Primary, CIS + AIS, AAH, Dysplasia, MIA\}
            \begin{itemize}
                \item Carcinoma in situ
                \item Adenocarcinoma in situ
                \item Atypical adenomatous hyperplasia
                \item Dysplasia
                \item Minimally invasive adenocarcinoma
            \end{itemize}
            \item Squamouse cell carcinoma (SQC) \& Adenocarcinoma (ADC)
            \begin{enumerate}
                \item Normal $\rightarrow$ Dysplasia $\rightarrow$ CIS $\rightarrow$ SQC (n=80)
                \item Normal $\rightarrow$ AAH $\rightarrow$ AIS $\rightarrow$ MIA $\rightarrow$ ADC (n=28)
            \end{enumerate}
        \end{itemize}
    \end{frame}

    \section{Methods}
    \subsection{Workflows}
    \begin{frame}
        \frametitle{Data pre-processing for variant discovery}

        \begin{figure}
            \includegraphics[width=0.3 \linewidth]{figures/Workflow/mapping.png}
            \caption{Data pre-processing for variant discovery \protect\cite{gatk1, gatk2}}
        \end{figure}
    \end{frame}

    \begin{frame}
        \frametitle{Somatic short variant discovery}

        \begin{figure}
            \includegraphics[width=0.7 \linewidth]{figures/Workflow/somatic_short_variants.png}
            \caption{Somatic short variant (SNVs + Indels) discovery workflow \protect\cite{gatk1, gatk2}}
        \end{figure}
    \end{frame}

    \begin{frame}
        \frametitle{Germline short variant discovery}

        \begin{figure}
            \includegraphics[width=0.7 \linewidth]{figures/Workflow/germline_short_variant.png}
            \caption{Germline short variant (SNVs + Indels) discovery workflow \protect\cite{gatk1, gatk2}}
        \end{figure}
    \end{frame}

    \begin{frame}
        \frametitle{RNA-seq short variant discovery}

        \begin{figure}
            \includegraphics[width=0.8 \linewidth]{figures/Workflow/RNA_short_variant.png}
            \caption{RNA-seq short variant (SNVs + Indels) discovery workflow \protect\cite{gatk1, gatk2}}
        \end{figure}
    \end{frame}

    \subsection{Miscellaneous}
    \begin{frame}
        \frametitle{Used Bioinformatics Tools}

        \begin{itemize}
            \item FastQC \cite{fastqc1}
            \item BWA \cite{bwa1, bwa2}
            \item STAR \cite{star1}
            \item Bowtie2 \cite{bowtie1}
            \item Samtools \cite{samtools1}
            \item GATK \cite{gatk1, gatk2}
            \item Picard \cite{picard1}
            \item VCF2MAF \cite{vcf2maf1}
            \item BCFtools \cite{bcftools1}
            \item VEP \cite{vep1}
            \item RSEM \cite{RSEM1}
            \item CIBERSORTx \cite{cibersort1}
        \end{itemize}
    \end{frame}

    \begin{frame}
        \frametitle{R Packages}

        \begin{itemize}
            \item Sequenza \cite{sequenza1}
            \item Copynumber \cite{copynumber1, copynumber2}
            \item DESeq2 \cite{DESeq1}
        \end{itemize}
    \end{frame}

    \begin{frame}
        \frametitle{Python Packages}

        \begin{itemize}
            \item Pandas \cite{pandas1, pandas2}
            \item Sequenza-utils \cite{sequenza1}
            \item Matplotlib \cite{matplotlib1}
            \item Seaborn \cite{seaborn1}
            \item CoMut \cite{comut1}
            \item PyClone \cite{pyclone1}
            \item Statannot
        \end{itemize}
    \end{frame}

    \section{Results}
    \subsection{Quality Checks with FastQC}
    \begin{frame}
        \frametitle{FastQC?}

        \begin{figure}
            \includegraphics[width=0.5 \linewidth]{figures/FastQC/example.png}
            \caption{Example of FastQC Result \protect\cite{fastqc1}}
        \end{figure}

        \begin{itemize}
            \item A quality check tool for sequence data
            \item Give an overview that which test may be problems
        \end{itemize}
    \end{frame}

    \begin{frame}
        \frametitle{FastQC on WES}

        \begin{figure}
            \includegraphics[width=\linewidth]{figures/FastQC/FastQC_WES.pdf}
            \caption{FastQC with WES data}
        \end{figure}

        $\therefore$ Only 33P1 has more than 3 failures: 6 FAILs. \\
        $\therefore$ 33P1 is excluded at further analysis.
    \end{frame}

    \begin{frame}[allowframebreaks]
        \frametitle{Failure on 33P1}

        \begin{figure}
            $\begin{array}{cc}
                \includegraphics[width=0.4 \linewidth]{figures/FastQC/33N.png}
                &
                \includegraphics[width=0.4 \linewidth]{figures/FastQC/33P1.png}
                \\
                \mbox{(a) 33N} & \mbox{(b) 33P1} \\
            \end{array}$
            \caption{Per Base Sequence Quality Results}
        \end{figure}

        \begin{figure}
            \includegraphics[width=\linewidth]{figures/FastQC/BWA.png}
            \caption{Coverage Depth Plot}
        \end{figure}
    \end{frame}

    \begin{frame}
        \frametitle{FastQC on WTS}

        \begin{figure}
            \includegraphics[width=\linewidth]{figures/FastQC/FastQC_WTS.pdf}
            \caption{FastQC with WTS data}
        \end{figure}

        $\therefore$ No sample has more than 5 failures. \\
        $\therefore$ All sample are good to analysis.
    \end{frame}

    \subsection{Quality Checks with Sequenza}
    \begin{frame}
        \frametitle{Sequenza?}

        \begin{figure}
            \includegraphics[width=0.6 \linewidth]{figures/Workflow/sequenza.jpg}
            \caption{Representative Output of the Sequenza \protect\cite{sequenza1}}
        \end{figure}
    \end{frame}

    \begin{frame}
        \frametitle{Cellularity \& Ploidy on WES}

        \begin{figure}
            $\begin{array}{cc}
                \includegraphics[width=0.4 \linewidth]{figures/Sequenza/BWA-sequenza-SQC.pdf}
                &
                \includegraphics[width=0.4 \linewidth]{figures/Sequenza/BWA-sequenza-ADC.pdf}
                \\
                \mbox{(a) SQC Samples} & \mbox{(b) ADC Samples} \\
            \end{array}$
            \caption{Cellularity and Ploidy from Sequenza}
        \end{figure}
    \end{frame}

    \begin{frame}
        \frametitle{Genome View on Patient \#57}

        \begin{figure}
            \includegraphics[width=\linewidth]{figures/Sequenza/57D1.pdf}
            \includegraphics[width=\linewidth]{figures/Sequenza/57C1.pdf}
            \includegraphics[width=\linewidth]{figures/Sequenza/57P1.pdf}
            \caption{Dysplasia-CIS-Primary tumor}
        \end{figure}
    \end{frame}

    \begin{frame}
        \frametitle{CNV of SQC}

        \begin{figure}
            \includegraphics[height=0.7 \textheight]{figures/Sequenza/BWA-genome-SQC.pdf}
            \caption{CNV Plot with SQC Patients}
        \end{figure}
    \end{frame}

    \begin{frame}
        \frametitle{CNV of ADC}

        \begin{figure}
            \includegraphics[height=0.7 \textheight]{figures/Sequenza/BWA-genome-ADC.pdf}
            \caption{CNV Plot with ADC Patients}
        \end{figure}
    \end{frame}

    \subsection{Find SNVs with Mutect2}
    \begin{frame}
        \frametitle{Mutect2?}

        \begin{figure}
            \includegraphics[width=0.6 \linewidth]{figures/Workflow/somatic_short_variants.png}
            \caption{Somatic short variant discovery workflow \protect\cite{gatk1, gatk2}}
        \end{figure}
    \end{frame}

    \begin{frame}
        \frametitle{Witer?}

        \begin{figure}
            \includegraphics[width=0.6 \linewidth]{figures/Workflow/witer.jpg}
            \caption{Witer diagram for detecting cancer-drive genes \protect\cite{witer1}}
        \end{figure}
    \end{frame}

    \begin{frame}
        \frametitle{Somatic Variant in SQC}

        \begin{figure}
            \includegraphics[width=\linewidth]{figures/Mutect2/Mutect2-SQC.pdf}
            \caption{CoMut Plot with SQC Patients}
        \end{figure}
    \end{frame}

    \begin{frame}
        \frametitle{Somatic Variant in ADC}

        \begin{figure}
            \includegraphics[width=\linewidth]{figures/Mutect2/Mutect2-ADC.pdf}
            \caption{CoMut Plot with ADC Patients}
        \end{figure}
    \end{frame}

    \subsection{Gene Expression Levels from RSEM}
    \begin{frame}
        \frametitle{RSEM?}

        \begin{figure}
            \includegraphics[width=0.2 \linewidth]{figures/Workflow/RSEM.jpg}
            \caption{The RSEM workflow \protect\cite{RSEM1}}
        \end{figure}
    \end{frame}

    \begin{frame}
        \frametitle{DEG Volcano Plots with Bowtie2 in ADC}

        Normal $\rightarrow$ Dysplasia $\rightarrow$ CIS $\rightarrow$ Primary (SQC)

        \begin{figure}
            $\begin{array}{ccc}
                \includegraphics[width=0.2 \linewidth]{figures/DEG/Volcano/SQC.Bowtie2.Normal-Dysplasia.volcano.pdf}
                &
                \includegraphics[width=0.2 \linewidth]{figures/DEG/Volcano/SQC.Bowtie2.Normal-CIS.volcano.pdf}
                &
                \includegraphics[width=0.2 \linewidth]{figures/DEG/Volcano/SQC.Bowtie2.Normal-Primary.volcano.pdf}
                \\
                \mbox{(a) Normal-Dysplasia} & \mbox{(b) Normal-CIS} & \mbox{(c) Normal-Primary} \\

                \includegraphics[width=0.2 \linewidth]{figures/DEG/Volcano/SQC.Bowtie2.Dysplasia-CIS.volcano.pdf}
                &
                \includegraphics[width=0.2 \linewidth]{figures/DEG/Volcano/SQC.Bowtie2.Dysplasia-Primary.volcano.pdf}
                &
                \includegraphics[width=0.2 \linewidth]{figures/DEG/Volcano/SQC.Bowtie2.CIS-Primary.volcano.pdf}
                \\
                \mbox{(d) Dysplasia-CIS} & \mbox{(e) Dysplasia-Primary} & \mbox{(f) CIS-Primary} \\
            \end{array}$
            \caption{DEG Volcano Plots with Bowtie2 in SQC}
        \end{figure}
    \end{frame}

    \begin{frame}
        \frametitle{DEG Pseudo-Venn Diagram with Bowtie2 in SQC}

         Normal $\rightarrow$ Dysplasia $\rightarrow$ CIS $\rightarrow$ Primary (SQC)

        \begin{figure}
            $\begin{array}{cc}
                \includegraphics[width=0.4 \linewidth]{figures/DEG/Venn/SQC.Up.Bowtie2.venn.pdf}
                &
                \includegraphics[width=0.4 \linewidth]{figures/DEG/Venn/SQC.Down.Bowtie2.venn.pdf}
                \\
                \mbox{(a) Up-regulated} & \mbox{(b) Down-regulated} \\
            \end{array}$
            \caption{DEG Pseudo-Venn Diagram with Bowtie2 in SQC}
        \end{figure}
    \end{frame}

    \begin{frame}
        \frametitle{DEG Venn Diagram with Bowtie2 in SQC}

        Normal $\rightarrow$ Dysplasia $\rightarrow$ CIS $\rightarrow$ Primary (SQC)

        \begin{figure}
            $\begin{array}{cc}
                \includegraphics[width=0.4 \linewidth]{figures/DEG/Pair-Venn/SQC.Up.Bowtie2.pair-venn.pdf}
                &
                \includegraphics[width=0.4 \linewidth]{figures/DEG/Pair-Venn/SQC.Down.Bowtie2.pair-venn.pdf}
                \\
                \mbox{(a) Up-regulated} & \mbox{(b) Down-regulated} \\
            \end{array}$
            \caption{DEG Venn Diagram with Bowtie2 in SQC}
        \end{figure}
    \end{frame}

    \begin{frame}
        \frametitle{DEG Volcano Plots with Bowtie2 in ADC}
        Normal $\rightarrow$ AAH $\rightarrow$ AIS $\rightarrow$ MIA $\rightarrow$ Primary (ADC)

        \begin{figure}
            $\begin{array}{ccc}
                \includegraphics[width=0.2 \linewidth]{figures/DEG/Volcano/ADC.Bowtie2.Normal-AAH.volcano.pdf}
                &
                \includegraphics[width=0.2 \linewidth]{figures/DEG/Volcano/ADC.Bowtie2.Normal-AIS.volcano.pdf}
                &
                \includegraphics[width=0.2 \linewidth]{figures/DEG/Volcano/ADC.Bowtie2.Normal-Primary.volcano.pdf}
                \\
                \mbox{(a) Normal-AAH} & \mbox{(b) Normal-AIS} & \mbox{(c) Normal-Primary} \\

                \includegraphics[width=0.2 \linewidth]{figures/DEG/Volcano/ADC.Bowtie2.AAH-AIS.volcano.pdf}
                &
                \includegraphics[width=0.2 \linewidth]{figures/DEG/Volcano/ADC.Bowtie2.AAH-Primary.volcano.pdf}
                &
                \includegraphics[width=0.2 \linewidth]{figures/DEG/Volcano/ADC.Bowtie2.AIS-Primary.volcano.pdf}
                \\
                \mbox{(d) AAH-AIS} & \mbox{(e) AAH-Primary} & \mbox{(f) AIS-Primary} \\
            \end{array}$
            \caption{DEG Volcano Plots with Bowtie2 in ADC}
        \end{figure}
    \end{frame}

    \begin{frame}
        \frametitle{DEG Pseudo-Venn Diagram with Bowtie2 in ADC}

        Normal $\rightarrow$ AAH $\rightarrow$ AIS $\rightarrow$ MIA $\rightarrow$ Primary (ADC)

        \begin{figure}
            $\begin{array}{cc}
                \includegraphics[width=0.4 \linewidth]{figures/DEG/Venn/ADC.Up.Bowtie2.venn.pdf}
                &
                \includegraphics[width=0.4 \linewidth]{figures/DEG/Venn/ADC.Down.Bowtie2.venn.pdf}
                \\
                \mbox{(a) Up-regulated} & \mbox{(b) Down-regulated} \\
            \end{array}$
            \caption{DEG Pseudo-Venn Diagram with Bowtie2 in ADC }
        \end{figure}
    \end{frame}

    \begin{frame}
        \frametitle{DEG Venn Diagram with Bowtie2 in ADC}

        Normal $\rightarrow$ AAH $\rightarrow$ AIS $\rightarrow$ MIA $\rightarrow$ Primary (ADC)

        \begin{figure}
            $\begin{array}{cc}
                \includegraphics[width=0.4 \linewidth]{figures/DEG/Pair-Venn/ADC.Up.Bowtie2.pair-venn.pdf}
                &
                \includegraphics[width=0.4 \linewidth]{figures/DEG/Pair-Venn/ADC.Down.Bowtie2.pair-venn.pdf}
                \\
                \mbox{(a) Up-regulated} & \mbox{(b) Down-regulated} \\
            \end{array}$
            \caption{DEG Venn Diagram with Bowtie2 in ADC}
        \end{figure}
    \end{frame}

    \subsection{Tumor Evolution with MesKit}
    \begin{frame}
        \frametitle{MesKit?}
    \end{frame}

    \subsection{Estimated Cell Types with CIBERSORTx}
    \begin{frame}
        \frametitle{CIBERSORTx?}

        \begin{figure}
            \includegraphics[width=0.6 \linewidth]{figures/Workflow/CIBERSORTx.png}
            \caption{CIBERSORTx workflow \protect\cite{cibersort1}}
        \end{figure}
    \end{frame}

    \begin{frame}
        \frametitle{CIBERSORTx with SQC}

        \begin{figure}
            \includegraphics[width=\linewidth]{figures/CIBERSORTx/SQC.Bowtie2.CibersortX.pdf}
            \caption{Estimated Cell Types with SQC Samples}
        \end{figure}
    \end{frame}

    \begin{frame}
        \frametitle{CIBERSORTx with ADC}

        \begin{figure}
            \includegraphics[width=0.6 \linewidth]{figures/CIBERSORTx/ADC.Bowtie2.CibersortX.pdf}
            \caption{Estimated Cell Types with ADC Samples}
        \end{figure}
    \end{frame}

    \section{Discussion}

    \begin{frame}[allowframebreaks]
        \frametitle{References}
        \bibliographystyle{apacite}
        \bibliography{reference}
    \end{frame}
\end{document}