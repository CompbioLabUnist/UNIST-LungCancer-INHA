% !TeX spellcheck = en_US
% !TeX encoding = UTF-8
\documentclass{beamer}

\mode<presentation> { \usetheme{Madrid} }

\usepackage{graphicx, graphics}
\usepackage[notocbib]{apacite}
\usepackage[style=iso]{datetime2}
\usepackage{enumerate}
\usepackage{multirow}
\usepackage{booktabs}
\DeclareGraphicsExtensions{.pdf, .png, .jpg, .gif}

\AtBeginSection[]
{
    \begin{frame}
        \vfill
        \centering
        \begin{beamercolorbox}[sep=8pt, center, shadow=true, rounded=true]{title}
            \usebeamerfont{title}
            \thesection. \insertsectionhead
            \par
        \end{beamercolorbox}
        \vfill
    \end{frame}
}

\AtBeginSubsection[]
{
    \begin{frame}
        \vfill
        \centering
        \begin{beamercolorbox}[sep=8pt, center, shadow=true, rounded=true]{title}
            \usebeamerfont{title}
            \thesection. \insertsectionhead
            \par
        \end{beamercolorbox}
        \begin{beamercolorbox}[sep=4pt, center, shadow=true, rounded=true]{title}
            \usebeamerfont{subtitle}
            \thesection.\thesubsection. \insertsubsectionhead
            \par
        \end{beamercolorbox}
        \vfill
    \end{frame}
}

\AtBeginSubsubsection[]
{
    \begin{frame}
        \vfill
        \centering
        \begin{beamercolorbox}[sep=8pt, center, shadow=true, rounded=true]{title}
            \usebeamerfont{title}
            \thesection. \insertsectionhead
            \par
        \end{beamercolorbox}
        \begin{beamercolorbox}[sep=4pt, center, shadow=true, rounded=true]{title}
            \usebeamerfont{subtitle}
            \thesection.\thesubsection. \insertsubsectionhead
            \par
        \end{beamercolorbox}
        \begin{beamercolorbox}[sep=3pt, center, shadow=true, rounded=true]{title}
            \usebeamerfont{subtitle}
            \thesection.\thesubsection.\thesubsubsection. \insertsubsubsectionhead
            \par
        \end{beamercolorbox}
        \vfill
    \end{frame}
}

\title[Lung Precancer]{Lung Precancer Study}

\author[Jaewoong Lee]
{
    Jaewoong Lee
    \and
    S. Park
    \and
    Y. Choi
    \and
    I. Yun
    \and
    Semin Lee
}

\institute[UNIST BME]
{
    Department of Biomedical Engineering
    \newline
    Ulsan National Institute of Science and Technology
    \medskip
    \newline
    \textit{jwlee230@unist.ac.kr}
}

\date{\today}

\begin{document}
    \begin{frame}
        \titlepage
    \end{frame}

    \begin{frame}
        \frametitle{Overview}
        \tableofcontents[hideallsubsections]
    \end{frame}

    \section{Introduction}
    \subsection{Lung Cancer}
    \begin{frame}
        \frametitle{Lung Cancer?}

        \begin{block}{The most common cancer}
            The most common form of cancer: \\
            12.3 \% of all cancers \cite{lung3}
        \end{block}

        \begin{block}{The most important factor}
            \textbf{Tobacco}
        \end{block}
    \end{frame}

    \begin{frame}
        \frametitle{Cancer Survival Rate in Korea}

        \begin{figure}
            \includegraphics[width=\linewidth]{figures/LungCancer/rate.png}
            \caption{Common cancer survival rates \protect\cite{lung6}}
        \end{figure}

        \begin{block}{Survival rate (More than 5 yr)}
            \begin{itemize}
                \item Tyroid: 68.4 \%
                \item Lung: 35.4 \%
            \end{itemize}
        \end{block}
    \end{frame}

    \begin{frame}
        \frametitle{Type of Lung Cancer}

        Types of lung cancer:
        \begin{enumerate}
            \item Adenocarcinoma (LUAD) (40 \%) $\bigstar$
            \item Squamous cell carcinoma (LUSC) (25 \%) $\bigstar$
            \item Small cell carcinoma (20 \%)
            \item Large cell carcinoma (10 \%)
            \item Adenosquamous carcinoma ($<$ 5 \%)
            \item Carcinoid ($<$ 5 \%)
            \item Bronchioalveolar (Bronchial gland carcinoma)
        \end{enumerate}
        \cite{lung1, lung2}
    \end{frame}

    \subsection{LUAD}
    \begin{frame}
        \frametitle{LUAD}
    \end{frame}

    \subsection{LUSC}
    \begin{frame}
        \frametitle{LUSC}
    \end{frame}

    \subsection{LUAD vs. LUSC}
    \begin{frame}[allowframebreaks]
        \frametitle{LUAD vs. LUSC}

        \begin{figure}
            $\begin{array}{cc}
                \includegraphics[width=0.3 \linewidth]{figures/LungCancer/ADC.png}
                &
                \includegraphics[width=0.3 \linewidth]{figures/LungCancer/SQC.png}
                \\
                \mbox{(a) LUAD} & \mbox{(b) LUSC} \\
            \end{array}$
            \caption{LUAD and LUSC histology in Lung cancer \cite{lung4}}
        \end{figure}

        \begin{figure}
            $\begin{array}{cc}
                \includegraphics[width=0.4 \linewidth]{figures/LungCancer/wang1.png}
                &
                \includegraphics[width=0.4 \linewidth]{figures/LungCancer/wang2.png}
                \\
                \mbox{(a) All patients} & \mbox{(b) By cancer stages} \\
            \end{array}$
            \caption{Kaplan-Meiere survival curves for LUAD \& LUSC \cite{lung5}}
        \end{figure}

        \begin{block}{Findings}
            LUSC is more dangerous than LUAD. $\because p < 0.001$
        \end{block}
    \end{frame}

    \subsection{Study Objectives}
    \begin{frame}
        \frametitle{Study Objectives}

        \begin{block}{Find different mutations}
            \begin{itemize}
                \item between WES vs. WTS
                \item from cancer vs. precancer
            \end{itemize}
        \end{block}

        \begin{block}{Pathway examine}
            \begin{itemize}
                \item with the mutation of WES \& RNA-seq
                \item with immune-depleted animal models
            \end{itemize}
        \end{block}

        \begin{block}{Ultra-deep sequencing}
            to find an \textit{infinitesimal} quantity of Non-Circulating Tumor DNA
            \begin{itemize}
                \item from blood
                \item from urine
                \item from bronchus
            \end{itemize}
        \end{block}
    \end{frame}

    \section{Materials}
    \begin{frame}
        \frametitle{Lung Cancer Data}

        \begin{itemize}
            \item Exome (n=289) + Transcriptome (n=166)
            \item Normal + \{Primary, CIS + AIS, AAH, Dysplasia, MIA\}
            \begin{itemize}
                \item Carcinoma in situ
                \item Adenocarcinoma in situ
                \item Atypical adenomatous hyperplasia
                \item Dysplasia
                \item Minimally invasive adenocarcinoma
            \end{itemize}
            \item Adenocarcinoma (LUAD) \& Squamouse cell carcinoma (LUSC)
            \begin{enumerate}
                \item Normal $\rightarrow$ AAH $\rightarrow$ AIS $\rightarrow$ MIA $\rightarrow$ LUAD (n=28)
                \item Normal $\rightarrow$ Dysplasia $\rightarrow$ CIS $\rightarrow$ LUSC (n=80)
            \end{enumerate}
        \end{itemize}
    \end{frame}

    \begin{frame}
        \frametitle{WES Data Composition}
    \end{frame}

    \begin{frame}
        \frametitle{WTS Data Composition}

        \begin{table}
            \caption{Number of WTS samples}
            \begin{tabular}{l|lr}
\toprule
     &       & Number of Samples \\
Cancer Subtype & Stage &                   \\
\midrule
\multirow{5}{*}{LUSC} & Normal &                17 \\
     & Dysplasia &                 2 \\
     & CIS+AIS &                34 \\
     & Primary &                36 \\
     & Total &                89 \\
\cline{1-3}
\multirow{5}{*}{LUAD} & Normal &                13 \\
     & AAH &                 1 \\
     & CIS+AIS &                 5 \\
     & Primary &                 6 \\
     & Total &                25 \\
\bottomrule
\end{tabular}

        \end{table}
    \end{frame}

    \section{Methods}
    \subsection{Workflows}
    \begin{frame}
        \frametitle{Data pre-processing for variant discovery}

        \begin{figure}
            \includegraphics[width=0.3 \linewidth]{figures/Workflow/mapping.png}
            \caption{Data pre-processing for variant discovery \protect\cite{gatk1, gatk2}}
        \end{figure}
    \end{frame}

    \begin{frame}
        \frametitle{Somatic short variant discovery}

        \begin{figure}
            \includegraphics[width=0.7 \linewidth]{figures/Workflow/somatic_short_variants.png}
            \caption{Somatic short variant (SNVs + Indels) discovery workflow \protect\cite{gatk1, gatk2}}
        \end{figure}
    \end{frame}

    \begin{frame}
        \frametitle{Germline short variant discovery}

        \begin{figure}
            \includegraphics[width=0.7 \linewidth]{figures/Workflow/germline_short_variant.png}
            \caption{Germline short variant (SNVs + Indels) discovery workflow \protect\cite{gatk1, gatk2}}
        \end{figure}
    \end{frame}

    \begin{frame}
        \frametitle{RNA-seq short variant discovery}

        \begin{figure}
            \includegraphics[width=0.8 \linewidth]{figures/Workflow/RNA_short_variant.png}
            \caption{RNA-seq short variant (SNVs + Indels) discovery workflow \protect\cite{gatk1, gatk2}}
        \end{figure}
    \end{frame}

    \section{Results}
    \subsection{Quality Checks}
    \begin{frame}
        \frametitle{FastQC?}

        \begin{figure}
            \includegraphics[width=0.5 \linewidth]{figures/Workflow/FastQC.png}
            \caption{Example of FastQC Result \protect\cite{fastqc1}}
        \end{figure}

        \begin{itemize}
            \item A quality check tool for sequence data
            \item Give an overview that which test may be problems
        \end{itemize}
    \end{frame}

    \begin{frame}
        \frametitle{FastQC on WES}

        \begin{figure}
            \includegraphics[width=\linewidth]{figures/FastQC/FastQC_WES.pdf}
            \caption{FastQC with WES data}
        \end{figure}

        \begin{alertblock}{Failure on 33P1 sample}
            33P1 is excluded at further analysis.
        \end{alertblock}
    \end{frame}

    \begin{frame}[allowframebreaks]
        \frametitle{Failure on 33P1}

        \begin{figure}
            $\begin{array}{cc}
                \includegraphics[width=0.4 \linewidth]{figures/FastQC/33N.png}
                &
                \includegraphics[width=0.4 \linewidth]{figures/FastQC/33P1.png}
                \\
                \mbox{(a) 33N} & \mbox{(b) 33P1} \\
            \end{array}$
            \caption{Per Base Sequence Quality Results}
        \end{figure}

        \begin{figure}
            \includegraphics[width=\linewidth]{figures/FastQC/BWA.png}
            \caption{Coverage Depth Plot}
        \end{figure}
    \end{frame}

    \begin{frame}
        \frametitle{FastQC on WTS}

        \begin{figure}
            \includegraphics[width=\linewidth]{figures/FastQC/FastQC_WTS.pdf}
            \caption{FastQC with WTS data}
        \end{figure}

        \begin{exampleblock}{All sample are good to analysis}
            $\because$ No sample has more than 5 failures.
        \end{exampleblock}
    \end{frame}

    \subsection{Copy Number Variations}
    \begin{frame}
        \frametitle{Sequenza?}

        \begin{figure}
            \includegraphics[width=0.6 \linewidth]{figures/Workflow/sequenza.jpg}
            \caption{Representative Output of the Sequenza \protect\cite{sequenza1}}
        \end{figure}
    \end{frame}

    \begin{frame}
        \frametitle{Cellularity \& Ploidy on WES}

        \begin{figure}
            $\begin{array}{cc}
                \includegraphics[width=0.4 \linewidth]{figures/Sequenza/BWA-sequenza-SQC.pdf}
                &
                \includegraphics[width=0.4 \linewidth]{figures/Sequenza/BWA-sequenza-ADC.pdf}
                \\
                \mbox{(a) LUSC Samples} & \mbox{(b) LUAD Samples} \\
            \end{array}$
            \caption{Cellularity and Ploidy from Sequenza}
        \end{figure}
    \end{frame}

    \begin{frame}
        \frametitle{Genome View on Patient \#57}

        \begin{figure}
            \includegraphics[width=\linewidth]{figures/Sequenza/57D1.pdf}
            \includegraphics[width=\linewidth]{figures/Sequenza/57C1.pdf}
            \includegraphics[width=\linewidth]{figures/Sequenza/57P1.pdf}
            \caption{Dysplasia-CIS-Primary Tumor on Patient \#57}
        \end{figure}
    \end{frame}

    \begin{frame}
        \frametitle{CNVs of LUSC}

        \begin{figure}
            \includegraphics[height=0.7 \textheight]{figures/Sequenza/BWA-genome-SQC.pdf}
            \caption{CNV Plot with LUSC Patients}
        \end{figure}
    \end{frame}

    \begin{frame}
        \frametitle{CNVs of LUAD}

        \begin{figure}
            \includegraphics[width=\linewidth]{figures/Sequenza/BWA-genome-ADC.pdf}
            \caption{CNV Plot with LUAD Patients}
        \end{figure}
    \end{frame}

    \begin{frame}
        \frametitle{LUSC vs. LUAD in CNV Plot}

        \begin{figure}
            \includegraphics[width=\linewidth]{figures/Sequenza/BWA-simple-SQC.pdf}
            \caption{Simple CNV Plot with LUSC Patients}
        \end{figure}

        \begin{figure}
            \includegraphics[width=\linewidth]{figures/Sequenza/BWA-simple-ADC.pdf}
            \caption{Simple CNV Plot with LUAD Patients}
        \end{figure}
    \end{frame}

    \begin{frame}
        \frametitle{Findings in Sequenza}
    \end{frame}

    \subsection{SNVs Analysis}
    \begin{frame}
        \frametitle{Mutect2?}

        \begin{figure}
            \includegraphics[width=0.6 \linewidth]{figures/Workflow/somatic_short_variants.png}
            \caption{Somatic short variant discovery workflow \protect\cite{gatk1, gatk2}}
        \end{figure}
    \end{frame}

    \begin{frame}
        \frametitle{MutEnricher?}

        \begin{figure}
            \includegraphics[width=0.8 \linewidth]{figures/Workflow/MutEnricher.jpg}
            \caption{Schematic representation of MunEnricher's analysis procedures \protect\cite{MutEnricher1}}
        \end{figure}
    \end{frame}

    \begin{frame}
        \frametitle{Driver Gene Selection Strategy}

        \begin{block}{COSMIC Cancer Gene Census \cite{CGC1}}
            Gene $\in$ CGC Tier 1 set
        \end{block}

        \begin{block}{Fisher FDR}
            Fisher FDR $ < 0.05$
        \end{block}

        \begin{block}{Fisher P-value}
            Fisher P-value $ < 0.05$
        \end{block}

        \begin{block}{Gene P-value}
            Gene P-value $ < 0.05 $
        \end{block}
    \end{frame}

    \begin{frame}
        \frametitle{Somatic Variant in LUSC}

        \begin{figure}
            \includegraphics[width=\linewidth]{figures/Mutect2/BWA-SQC.pdf}
            \caption{CoMut Plot with LUSC Patients}
        \end{figure}
    \end{frame}

    \begin{frame}
        \frametitle{Somatic Variant in LUAD}

        \begin{figure}
            \includegraphics[width=\linewidth]{figures/Mutect2/BWA-ADC.pdf}
            \caption{CoMut Plot with LUAD Patients}
        \end{figure}
    \end{frame}

    \begin{frame}
        \frametitle{Findings in SNVs Analysis}
    \end{frame}

    \subsection{VAF Analysis}
    \begin{frame}
        \frametitle{VAF?}

        \begin{itemize}
            \item Variant allele frequency
            \item $\textrm{VAF} = \textrm{Alternative allele read count} / \textrm{Total read count}$
            \item To find tumor evolution
        \end{itemize}

        \begin{figure}
            \includegraphics[width=0.4 \linewidth]{figures/VAF/VAF.jpg}
            \caption{VAF distribution of validated mutations \protect\cite{VAF1}}
        \end{figure}
    \end{frame}

    \begin{frame}[allowframebreaks]
        \frametitle{VAF Plots}
    \end{frame}

    \begin{frame}
        \frametitle{PyClone?}

        \begin{figure}
            \includegraphics[width=0.8 \linewidth]{figures/Workflow/PyClone.jpg}
            \caption{Analysis of multiple samples by PyClone \protect\cite{pyclone1}}
        \end{figure}
    \end{frame}

    \begin{frame}[allowframebreaks]
        \frametitle{PyClone Plots}
    \end{frame}

    \begin{frame}
        \frametitle{Findings in VAF Analysis}
    \end{frame}

    \subsection{Tumor Evolution Trajectories Analysis}
    \begin{frame}
        \frametitle{Revolver?}

        \begin{figure}
            \includegraphics[width=0.8 \linewidth]{figures/Workflow/revolver.jpg}
            \caption{Repeated Evolutionary Trajectories \protect\cite{revolver1}}
        \end{figure}
    \end{frame}

    \begin{frame}
        \frametitle{Findings in Tumor Evolution Trajectories Analysis}
    \end{frame}

    \subsection{Differences in Gene Expression Levels}
    \begin{frame}
        \frametitle{RSEM?}

        \begin{figure}
            \includegraphics[width=0.2 \linewidth]{figures/Workflow/RSEM.jpg}
            \caption{RSEM workflow \protect\cite{RSEM1}}
        \end{figure}
    \end{frame}

    \begin{frame}
        \frametitle{DESeq2?}

        \begin{figure}
            \includegraphics[width=0.5 \linewidth]{figures/Workflow/DESeq2.png}
            \caption{DESeq2 workflow \protect\cite{DESeq1}}
        \end{figure}
    \end{frame}

    \begin{frame}
        \frametitle{DEG Selection Strategy}
        DEG: differentially expressed genes

        \begin{block}{Fold Change}
            $\log_{2}(\text{Fold Change}) > 1 \vee \log_{2}(\text{Fold Change}) < -1$
        \end{block}

        \begin{block}{P-value}
            $\text{P-value} < 0.05$
        \end{block}

        \begin{block}{Adjusted P-value}
            $\text{Padj} < 0.05$
        \end{block}
    \end{frame}

    \begin{frame}
        \frametitle{Enrichr?}

        \begin{figure}
            \includegraphics[width=0.8 \linewidth]{figures/Workflow/Enrichr.jpg}
            \caption{Enrichr workflow \protect\cite{Enrichr1, Enrichr2}}
        \end{figure}
    \end{frame}

    \begin{frame}
        \frametitle{Gene-set Library}

        \begin{figure}
            \includegraphics[width=0.6 \linewidth]{figures/Workflow/KEGG.jpg}
            \caption{The global map of metabolic pathways by KEGG \protect\cite{KEGG1}}
        \end{figure}

        \begin{block}{KEGG}
            KEGG 2021 Human
        \end{block}
    \end{frame}

    \begin{frame}[allowframebreaks]
        \frametitle{WTS Data Composition}

        \begin{table}
            \caption{Number of WTS samples}
            \begin{tabular}{l|lr}
\toprule
     &       & Number of Samples \\
Cancer Subtype & Stage &                   \\
\midrule
\multirow{5}{*}{LUSC} & Normal &                17 \\
     & Dysplasia &                 2 \\
     & CIS+AIS &                34 \\
     & Primary &                36 \\
     & Total &                89 \\
\cline{1-3}
\multirow{5}{*}{LUAD} & Normal &                13 \\
     & AAH &                 1 \\
     & CIS+AIS &                 5 \\
     & Primary &                 6 \\
     & Total &                25 \\
\bottomrule
\end{tabular}

        \end{table}

        \begin{table}
            \caption{Number of WTS LUSC samples}
            \begin{tabular}{l|lr}
Recurrence? & Stage & Number of Samples \\ \hline
\multirow{4}{*}{Recurrence (n=13)} & Normal & 1 \\
 & Dysplasia & 1 \\
 & CIS & 5 \\
 & Primary & 6 \\ \hline
\multirow{4}{*}{Non-recurrence (n=74)} & Normal & 16 \\
 & Dysplasia & 1 \\
 & CIS & 28 \\
 & Primary & 29
\end{tabular}

        \end{table}

        \begin{table}
            \caption{Number of WTS LUAD samples}
            \begin{tabular}{l|lr}
Recurrence? & Stage & Number of samples \\ \hline
\multirow{5}{*}{Recurrence (n=4)} & Normal & 1 \\
 & AAH & 0 \\
 & AIS & 2 \\
 & MIA & 0 \\
 & Primary & 1 \\ \hline
\multirow{5}{*}{Non-recurrence (n=26)} & Normal & 11 \\
 & AAH & 1 \\
 & AIS & 7 \\
 & MIA & 0 \\
 & Primary & 7
\end{tabular}
        \end{table}
    \end{frame}

    \subsubsection{Comparing cancer stage in LUSC}
    \begin{frame}
        \frametitle{DEG List in LUSC}

        \begin{table}
            \caption{Up-regulated DEG in LUSC}
            \begin{tabular}{lrrr}
\toprule
   gene &  log2FoldChange &   pvalue &     padj \\
\midrule
 AKR1C1 &        6.18e+00 & 5.14e-26 & 5.01e-23 \\
 AKR1C2 &        6.06e+00 & 1.19e-22 & 5.04e-20 \\
CYP4F11 &        5.58e+00 & 1.51e-20 & 4.36e-18 \\
\bottomrule
\end{tabular}

        \end{table}

        \begin{table}
            \caption{Down-regulated DEG in LUSC}
            \begin{tabular}{lrrr}
\toprule
   gene &  log2FoldChange &   pvalue &     padj \\
\midrule
  SFTPC &       -5.85e+00 & 9.16e-21 & 2.83e-18 \\
FAM107A &       -4.62e+00 & 2.27e-33 & 9.60e-30 \\
 LRRC36 &       -4.53e+00 & 5.49e-36 & 3.48e-32 \\
\bottomrule
\end{tabular}

        \end{table}
    \end{frame}

    \begin{frame}
        \frametitle{DEG Volcano Plots in LUSC}
        Normal $\rightarrow$ Dysplasia $\rightarrow$ CIS $\rightarrow$ Primary (LUSC)

        \begin{figure}
            $\begin{array}{ccc}
                \includegraphics[width=0.2 \linewidth]{figures/DEG/Volcano/STAR.SQC.Normal-Dysplasia.volcano.pdf}
                &
                \includegraphics[width=0.2 \linewidth]{figures/DEG/Volcano/STAR.SQC.Normal-CIS.volcano.pdf}
                &
                \includegraphics[width=0.2 \linewidth]{figures/DEG/Volcano/STAR.SQC.Normal-Primary.volcano.pdf}
                \\
                \mbox{(a) Normal-Dysplasia} & \mbox{(b) Normal-CIS} & \mbox{(c) Normal-Primary} \\

                \includegraphics[width=0.2 \linewidth]{figures/DEG/Volcano/STAR.SQC.Dysplasia-CIS.volcano.pdf}
                &
                \includegraphics[width=0.2 \linewidth]{figures/DEG/Volcano/STAR.SQC.Dysplasia-Primary.volcano.pdf}
                &
                \includegraphics[width=0.2 \linewidth]{figures/DEG/Volcano/STAR.SQC.CIS-Primary.volcano.pdf}
                \\
                \mbox{(d) Dysplasia-CIS} & \mbox{(e) Dysplasia-Primary} & \mbox{(f) CIS-Primary} \\
            \end{array}$
            \caption{DEG Volcano Plots in LUSC}
        \end{figure}
    \end{frame}

    \begin{frame}
        \frametitle{DEG Venn Diagram in LUSC}
        Normal $\rightarrow$ Dysplasia $\rightarrow$ CIS $\rightarrow$ Primary (LUSC)

        \begin{figure}
            $\begin{array}{ccc}
                \includegraphics[width=0.3 \linewidth]{figures/DEG/Pair-Venn/STAR.SQC.Up.venn.pdf}
                &
                \includegraphics[width=0.3 \linewidth]{figures/DEG/Pair-Venn/STAR.SQC.venn.pdf}
                &
                \includegraphics[width=0.3 \linewidth]{figures/DEG/Pair-Venn/STAR.SQC.Down.venn.pdf}
                \\
                \mbox{(a) Up-regulated} & \mbox{(b) Both}& \mbox{(c) Down-regulated} \\
            \end{array}$
            \caption{DEG Venn Diagram in LUSC}
        \end{figure}
    \end{frame}

    \begin{frame}
        \frametitle{Enrichment test with Normal vs. Dysplasia in LUSC}

        \begin{table}
            \caption{Up-regulated Pathways on Normal vs. Dysplasia}
            \resizebox{\linewidth}{!}
            {\begin{tabular}{lr}
\toprule
                 Term name &  Adjusted p-value \\
\midrule
             Leishmaniasis &          6.72e-03 \\
                  Lysosome &          6.72e-03 \\
                 Phagosome &          1.15e-02 \\
 Th17 cell differentiation &          1.29e-02 \\
             Toxoplasmosis &          1.29e-02 \\
\bottomrule
\end{tabular}
}
        \end{table}

        \begin{table}
            \caption{Down-regulated Pathways on Normal vs. Dysplasia}
            \resizebox{\linewidth}{!}
            {\input{tables/RSEM/STAR.SQC.Normal-Dysplasia.Down.KEGG.tex}}
        \end{table}
    \end{frame}

    \begin{frame}
        \frametitle{Enrichment test with Normal vs. CIS in LUSC}

        \begin{table}
            \caption{Up-regulated Pathways on Normal vs. CIS}
            \resizebox{\linewidth}{!}
            {\begin{tabular}{lr}
\toprule
                   Term name &  Adjusted p-value \\
\midrule
  Hematopoietic cell lineage &          7.22e-08 \\
                     Malaria &          1.16e-06 \\
     Cell adhesion molecules &          1.16e-06 \\
 Hypertrophic cardiomyopathy &          1.16e-06 \\
   Th17 cell differentiation &          2.34e-06 \\
\bottomrule
\end{tabular}
}
        \end{table}

        \begin{table}
            \caption{Down-regulated Pathways on Normal vs. CIS}
            \resizebox{\linewidth}{!}
            {\begin{tabular}{lr}
\toprule
                                   Term name &  Adjusted p-value \\
\midrule
Metabolism of xenobiotics by cytochrome P450 &          9.34e-06 \\
                             Drug metabolism &          9.06e-05 \\
                                  Cell cycle &          1.68e-04 \\
\bottomrule
\end{tabular}
}
        \end{table}
    \end{frame}

    \begin{frame}
        \frametitle{Enrichment test with Normal vs. Primary in LUSC}

        \begin{table}
            \caption{Up-regulated Pathways on Normal vs. Primary}
            \resizebox{\linewidth}{!}
            {\begin{tabular}{lr}
\toprule
                                   Term name &  Adjusted p-value \\
\midrule
                Glycolysis / Gluconeogenesis &          1.88e-06 \\
Metabolism of xenobiotics by cytochrome P450 &          5.19e-06 \\
                      Glutathione metabolism &          5.77e-06 \\
                             Drug metabolism &          5.77e-06 \\
         Central carbon metabolism in cancer &          5.77e-06 \\
\bottomrule
\end{tabular}
}
        \end{table}

        \begin{table}
            \caption{Down-regulated Pathways on Normal vs. Primary}
            \resizebox{\linewidth}{!}
            {\begin{tabular}{lr}
\toprule
                  Term name &  Adjusted p-value \\
\midrule
 Hematopoietic cell lineage &          1.84e-10 \\
    Cell adhesion molecules &          3.10e-07 \\
Hypertrophic cardiomyopathy &          3.10e-07 \\
                    Malaria &          3.10e-07 \\
          Viral myocarditis &          2.04e-05 \\
\bottomrule
\end{tabular}
}
        \end{table}
    \end{frame}

    \begin{frame}
        \frametitle{Findings in Comparing cancer stage in LUSC}

        \begin{block}{AKR1C1 \& AKR1C2}
            \begin{enumerate}
                \item They are down-regulated in CIS, but up-regulated in Primary.
                \item They regulate steroids \cite{AKR1C1-1} and hormones \cite{AKR1C1-2} .
                \item They promote the metastasis of NSCLC \cite{AKR1C1-3}.
            \end{enumerate}
        \end{block}

        \begin{block}{Down-regulation of SFTPC}
            \begin{enumerate}
                \item A pulmonary surfactant associated protein \cite{SFTPC4}.
                \item SFTPC $\Downarrow$ $\Rightarrow$ Poor survival in LUAD \cite{SFTPC1}.
                \item Associated with lung disease in adult \cite{SFTPC2} and baby \cite{SFTPC3}.
            \end{enumerate}
        \end{block}
    \end{frame}

    \subsubsection{Comparing cancer stage in LUAD}
    \begin{frame}
        \frametitle{DEG List in LUSC}

        \begin{table}
            \caption{Up-regulated DEG in LUAD}
            \begin{tabular}{lrrr}
\toprule
 gene &  log2FoldChange &   pvalue &     padj \\
\midrule
ABCA4 &        4.95e+00 & 3.01e-12 & 2.58e-09 \\
HMGA2 &        4.79e+00 & 8.06e-08 & 1.46e-05 \\
KIF12 &        4.48e+00 & 1.33e-06 & 1.46e-04 \\
\bottomrule
\end{tabular}

        \end{table}

        \begin{table}
            \caption{Down-regulated DEG in LUAD}
            \begin{tabular}{lrrr}
\toprule
  gene &  log2FoldChange &   pvalue &     padj \\
\midrule
SLC6A4 &       -6.20e+00 & 5.80e-10 & 2.36e-07 \\
IL1RL1 &       -4.20e+00 & 7.47e-06 & 5.82e-04 \\
RNF185 &       -4.06e+00 & 4.75e-05 & 2.45e-03 \\
\bottomrule
\end{tabular}

        \end{table}
    \end{frame}

    \begin{frame}
        \frametitle{DEG Volcano Plots in LUAD}
        Normal $\rightarrow$ AAH $\rightarrow$ AIS $\rightarrow$ Primary (LUAD)

        \begin{figure}
            $\begin{array}{ccc}
                \includegraphics[width=0.2 \linewidth]{figures/DEG/Volcano/STAR.ADC.Normal-AAH.volcano.pdf}
                &
                \includegraphics[width=0.2 \linewidth]{figures/DEG/Volcano/STAR.ADC.Normal-AIS.volcano.pdf}
                &
                \includegraphics[width=0.2 \linewidth]{figures/DEG/Volcano/STAR.ADC.Normal-Primary.volcano.pdf}
                \\
                \mbox{(a) Normal-AAH} & \mbox{(b) Normal-AIS} & \mbox{(c) Normal-Primary} \\

                \includegraphics[width=0.2 \linewidth]{figures/DEG/Volcano/STAR.ADC.AAH-AIS.volcano.pdf}
                &
                \includegraphics[width=0.2 \linewidth]{figures/DEG/Volcano/STAR.ADC.AAH-Primary.volcano.pdf}
                &
                \includegraphics[width=0.2 \linewidth]{figures/DEG/Volcano/STAR.ADC.AIS-Primary.volcano.pdf}
                \\
                \mbox{(d) AAH-AIS} & \mbox{(e) AAH-Primary} & \mbox{(f) AIS-Primary} \\
            \end{array}$
            \caption{DEG Volcano Plots in LUAD}
        \end{figure}
    \end{frame}

    \begin{frame}
        \frametitle{DEG Venn Diagram in LUAD}

        Normal $\rightarrow$ AAH $\rightarrow$ AIS $\rightarrow$ Primary (LUAD)

        \begin{figure}
            $\begin{array}{ccc}
                \includegraphics[width=0.3 \linewidth]{figures/DEG/Pair-Venn/STAR.ADC.Up.venn.pdf}
                &
                \includegraphics[width=0.3 \linewidth]{figures/DEG/Pair-Venn/STAR.ADC.venn.pdf}
                &
                \includegraphics[width=0.3 \linewidth]{figures/DEG/Pair-Venn/STAR.ADC.Down.venn.pdf}
                \\
                \mbox{(a) Up-regulated} &\mbox{(b) Both} & \mbox{(c) Down-regulated} \\
            \end{array}$
            \caption{DEG Venn Diagram in LUAD}
        \end{figure}
    \end{frame}

    \begin{frame}
        \frametitle{Enrichment test with Normal vs. AAH in LUAD}

        \begin{table}
            \caption{Up-regulated Pathways on Normal vs. AAH}
            \resizebox{\linewidth}{!}
            {\input{tables/RSEM/STAR.ADC.Normal-AAH.Up.KEGG.tex}}
        \end{table}

        \begin{table}
            \caption{Down-regulated Pathways on Normal vs. AAH}
            \resizebox{\linewidth}{!}
            {\input{tables/RSEM/STAR.ADC.Normal-AAH.Down.KEGG.tex}}
        \end{table}
    \end{frame}

    \begin{frame}
        \frametitle{Enrichment test with Normal vs. AIS in LUAD}

        \begin{table}
            \caption{Up-regulated Pathways on Normal vs. AIS}
            \resizebox{\linewidth}{!}
            {\begin{tabular}{lr}
\toprule
                 Term name &  Adjusted p-value \\
\midrule
 Calcium signaling pathway &          2.49e-02 \\
   Cell adhesion molecules &          3.55e-02 \\
\bottomrule
\end{tabular}
}
        \end{table}

        \begin{table}
            \caption{Down-regulated Pathways on Normal vs. AIS}
            \resizebox{\linewidth}{!}
            {\input{tables/RSEM/STAR.ADC.Normal-AIS.Down.KEGG.tex}}
        \end{table}
    \end{frame}

    \begin{frame}
        \frametitle{Enrichment test with Normal vs. Primary in LUAD}

        \begin{table}
            \caption{Up-regulated Pathways on Normal vs. Primary}
            \resizebox{\linewidth}{!}
            {\input{tables/RSEM/STAR.ADC.Normal-Primary.Up.KEGG.tex}}
        \end{table}

        \begin{table}
            \caption{Down-regulated Pathways on Normal vs. Primary}
            \resizebox{\linewidth}{!}
            {\begin{tabular}{lr}
\toprule
                         Term name &  Adjusted p-value \\
\midrule
                    Focal adhesion &          2.43e-03 \\
          ECM-receptor interaction &          3.46e-03 \\
         Relaxin signaling pathway &          1.34e-02 \\
Vascular smooth muscle contraction &          1.34e-02 \\
       Hypertrophic cardiomyopathy &          1.34e-02 \\
\bottomrule
\end{tabular}
}
        \end{table}
    \end{frame}

    \begin{frame}[allowframebreaks]
        \frametitle{Finding in Comparing cancer stage in LUAD}

        \begin{block}{ABCA4}
            \begin{enumerate}
                \item Down-regulated in AAH \& AIS, but up-regulated in Primary.
                \item It is associated with ophthalmology \cite{ABCA4-2}.
                \item It shows lung cancer susceptibility in Korean patients \cite{ABCA4-1}.
            \end{enumerate}
        \end{block}

        \begin{block}{KCNQ3}
            \begin{enumerate}
                \item Down-regulated in AIS, but up-regulated in Primary.
                \item K$^+$ voltage-dependent channels $ \Rightarrow $ Various physiological functions \cite{KCNQ3-1, KCNQ3-2, KCNQ3-3}.
                \item Up-regulated microRNAs in hypoxia-induced LUAD \cite{KCNQ3-4}.
                \item KCNQ gene family is associated with lung disease \cite{KCNQ3-5}.
            \end{enumerate}
        \end{block}

        \begin{block}{CHRM1}
            \begin{enumerate}
                \item Up-regulated in AIS, but down-regulated in Primary.
                \item Various cellular responses $\Rightarrow$ neurodevelopmental disorders \cite{CHRM1-1}, schizophrenia \cite{CHRM1-2}, and Alzheimer's disease \cite{CHRM1-3}.
                \item Reported down-regulation in LUSC \& LUAD \cite{CHRM1-4}.
            \end{enumerate}
        \end{block}
    \end{frame}

    \subsubsection{Recur vs. Non-recur in LUSC}
    \begin{frame}
        \frametitle{LUSC Data Composition}

        \begin{table}
            \caption{Number of WTS LUSC samples}
            \begin{tabular}{l|lr}
Recurrence? & Stage & Number of Samples \\ \hline
\multirow{4}{*}{Recurrence (n=13)} & Normal & 1 \\
 & Dysplasia & 1 \\
 & CIS & 5 \\
 & Primary & 6 \\ \hline
\multirow{4}{*}{Non-recurrence (n=74)} & Normal & 16 \\
 & Dysplasia & 1 \\
 & CIS & 28 \\
 & Primary & 29
\end{tabular}

        \end{table}

        \begin{block}{Pooled normal samples}
            In order to compare with Normal stage, merging Normal samples. \\
            $\because$ Insufficient number of Normal samples in Recur.
        \end{block}
    \end{frame}

    \begin{frame}
        \frametitle{DEG Volcano Plots for R vs. NR with CIS in LUSC}

        \begin{figure}
            $\begin{array}{cc}
                \includegraphics[width=0.4 \linewidth]{figures/DEG/Volcano/STAR.SQC.Recur.Normal-CIS.volcano.pdf}
                &
                \includegraphics[width=0.4 \linewidth]{figures/DEG/Volcano/STAR.SQC.Nonrecur.Normal-CIS.volcano.pdf}
                \\
                \mbox{(a) Recur} & \mbox{(b) Non-recur} \\
            \end{array}$
            \caption{DEG Volcanot Plot with CIS in LUSC}
        \end{figure}
    \end{frame}

    \begin{frame}
        \frametitle{DEG Venn Diagram for R vs. NR with CIS in  LUSC}

        \begin{figure}
            $\begin{array}{ccc}
                \includegraphics[width=0.3 \linewidth]{figures/DEG/Pair-Venn/STAR.SQC-CIS.Recur-Nonrecur.Up.venn.pdf}
                &
                \includegraphics[width=0.3 \linewidth]{figures/DEG/Pair-Venn/STAR.SQC-CIS.Recur-Nonrecur.venn.pdf}
                &
                \includegraphics[width=0.3 \linewidth]{figures/DEG/Pair-Venn/STAR.SQC-CIS.Recur-Nonrecur.Down.venn.pdf}
                \\
                \mbox{(a) Up-regulated} & \mbox{(b) Both} & \mbox{(c) Down-regulated} \\
            \end{array}$
            \caption{DEG Venn Diagram for R vs. NR with CIS in LUSC}
        \end{figure}
    \end{frame}

    \begin{frame}
        \frametitle{Enrichment test for Recur-specific with CIS in LUSC}

        \begin{table}
            \caption{Up-regulated Pathways for Recur-specific}
            \resizebox{\linewidth}{!}
            {\begin{tabular}{llr}
\toprule
                                      Term name &      Overlapping genes... &  Adjusted p-value \\
\midrule
                         Dilated cardiomyopathy &  CACNG8,CACNB4,PLN,...(7) &          4.70e-02 \\
Arrhythmogenic right ventricular cardiomyopathy & CACNG8,CACNB4,SGCB,...(6) &          4.70e-02 \\
\bottomrule
\end{tabular}
}
        \end{table}

        \begin{table}
            \caption{Down-regulated Pathways for Recur-specific}
            \resizebox{\linewidth}{!}
            {\begin{tabular}{llr}
\toprule
                    Term name &        Overlapping genes... &  Adjusted p-value \\
\midrule
           Huntington disease &   COX8A,DCTN5,COX7B,...(24) &          6.36e-06 \\
Amyotrophic lateral sclerosis &  DCTN5,COX7B,TOMM40,...(25) &          1.62e-05 \\
            Parkinson disease & COX8A,COX7B,NDUFA12,...(20) &          1.62e-05 \\
\bottomrule
\end{tabular}
}
        \end{table}
    \end{frame}

    \begin{frame}
        \frametitle{Enrichment test for Non-recur-specific with CIS in LUSC}

        \begin{table}
            \caption{Up-regulated Pathways for Non-recur-specific}
            \resizebox{\linewidth}{!}
            {\begin{tabular}{llr}
\toprule
                 Term name (28) &          Overlapping genes... &  Adjusted p-value \\
\midrule
     Hematopoietic cell lineage &      IL6,HLA-DMA,IL4R,...(12) &          3.50e-05 \\
     Inflammatory bowel disease &      HLA-DMA,IL6,IL4R,...(10) &          3.50e-05 \\
Staphylococcus aureus infection & HLA-DMA,SELPLG,FCGR2A,...(11) &          1.13e-04 \\
\bottomrule
\end{tabular}
}
        \end{table}

        \begin{table}
            \caption{Down-regulated Pathways for Non-recur-specific}
            \resizebox{\linewidth}{!}
            {\begin{tabular}{llr}
\toprule
            Term name (5) &       Overlapping genes... &  Adjusted p-value \\
\midrule
Citrate cycle (TCA cycle) &    PDHA1,MDH1,MDH2,...(10) &          5.82e-05 \\
          Drug metabolism &  GSTM4,GSTM3,GSTM2,...(16) &          1.39e-03 \\
            RNA transport & EIF2B5,EIF5B,PRMT5,...(20) &          9.42e-03 \\
\bottomrule
\end{tabular}
}
        \end{table}
    \end{frame}

    \begin{frame}
        \frametitle{Enrichment test for Intersected with CIS in LUSC}

        \begin{table}
            \caption{Up-regulated Pathways for Intersected}
            \resizebox{\linewidth}{!}
            {\begin{tabular}{llr}
\toprule
             Term name (41) &     Overlapping genes... &  Adjusted p-value \\
\midrule
Hypertrophic cardiomyopathy &  EDN1,CACNB4,ACE,...(12) &          1.44e-05 \\
    Cell adhesion molecules & CADM1,ICAM2,SELP,...(15) &          1.44e-05 \\
 Hematopoietic cell lineage & CSF3,HLA-DMA,MME,...(12) &          2.57e-05 \\
\bottomrule
\end{tabular}
}
        \end{table}

        \begin{table}
            \caption{Down-regulated Pathways for Intersected}
            \resizebox{\linewidth}{!}
            {\input{tables/RSEM/STAR.SQC-CIS.both.Down.KEGG.tex}}
        \end{table}
    \end{frame}

    \begin{frame}
        \frametitle{DEG Volcano Plots for R vs. NR with Primary in LUSC}

        \begin{figure}
            $\begin{array}{cc}
                \includegraphics[width=0.4 \linewidth]{figures/DEG/Volcano/STAR.SQC.Recur.Normal-Primary.volcano.pdf}
                &
                \includegraphics[width=0.4 \linewidth]{figures/DEG/Volcano/STAR.SQC.Nonrecur.Normal-Primary.volcano.pdf}
                \\
                \mbox{(a) Recur} & \mbox{(b) Non-recur} \\
            \end{array}$
            \caption{DEG Volcanot Plot with Primary in LUSC}
        \end{figure}
    \end{frame}

     \begin{frame}
        \frametitle{DEG Venn Diagram for R vs. NR with Primary in  LUSC}

        \begin{figure}
            $\begin{array}{ccc}
                \includegraphics[width=0.3 \linewidth]{figures/DEG/Pair-Venn/STAR.SQC-Primary.Recur-Nonrecur.Up.venn.pdf}
                &
                \includegraphics[width=0.3 \linewidth]{figures/DEG/Pair-Venn/STAR.SQC-Primary.Recur-Nonrecur.venn.pdf}
                &
                \includegraphics[width=0.3 \linewidth]{figures/DEG/Pair-Venn/STAR.SQC-Primary.Recur-Nonrecur.Down.venn.pdf}
                \\
                \mbox{(a) Up-regulated} & \mbox{(b) Both} & \mbox{(c) Down-regulated} \\
            \end{array}$
            \caption{DEG Venn Diagram for R vs. NR with Primary in LUSC}
        \end{figure}
    \end{frame}

    \begin{frame}
        \frametitle{Enrichment test for Recur-specific with Primary in LUSC}

        \begin{table}
            \caption{Up-regulated Pathways for Recur-specific}
            \resizebox{\linewidth}{!}
            {\begin{tabular}{lr}
\toprule
                    Term name &  Adjusted p-value \\
\midrule
Amyotrophic lateral sclerosis &          4.85e-03 \\
                RNA transport &          6.11e-03 \\
    mRNA surveillance pathway &          6.11e-03 \\
\bottomrule
\end{tabular}
}
        \end{table}

        \begin{table}
            \caption{Down-regulated Pathways for Recur-specific}
            \resizebox{\linewidth}{!}
            {\input{tables/RSEM/STAR.SQC-Primary.Recur.Down.KEGG.tex}}
        \end{table}
    \end{frame}

    \begin{frame}
        \frametitle{Enrichment test for NR-specific with Primary in LUSC}

        \begin{table}
            \caption{Up-regulated Pathways for Non-recur-specific}
            \resizebox{\linewidth}{!}
            {\begin{tabular}{llr}
\toprule
               Term name &     Overlapping genes... &  Adjusted p-value \\
\midrule
Homologous recombination & RAD51D,POLD1,RPA3,...(9) &          1.00e-02 \\
\bottomrule
\end{tabular}
}
        \end{table}

        \begin{table}
            \caption{Down-regulated Pathways for Non-recur-specific}
            \resizebox{\linewidth}{!}
            {\begin{tabular}{llr}
\toprule
                      Term name & Overlapping genes... &  Adjusted p-value \\
\midrule
Staphylococcus aureus infection &   IL10,CFD,ITGB2,... &          5.37e-05 \\
     Hematopoietic cell lineage &    CR1,MME,ITGB3,... &          5.37e-05 \\
                  Leishmaniasis &   IL10,C3,NFKBIA,... &          4.30e-04 \\
\bottomrule
\end{tabular}
}
        \end{table}
    \end{frame}

    \begin{frame}
        \frametitle{Enrichment test for Intersected with Primary in LUSC}

        \begin{table}
            \caption{Up-regulated Pathways for Intersected}
            \resizebox{\linewidth}{!}
            {\begin{tabular}{llr}
\toprule
                                   Term name &      Overlapping genes... &  Adjusted p-value \\
\midrule
                Glycolysis / Gluconeogenesis &    GPI,TPI1,PDHA1,...(16) &          1.09e-06 \\
                             Drug metabolism & GSTM4,GSTM3,GSTM2,...(20) &          1.09e-06 \\
Metabolism of xenobiotics by cytochrome P450 &  GSTM4,CBR1,GSTM3,...(15) &          2.27e-05 \\
\bottomrule
\end{tabular}
}
        \end{table}

        \begin{table}
            \caption{Down-regulated Pathways for Intersected}
            \resizebox{\linewidth}{!}
            {\input{tables/RSEM/STAR.SQC-Primary.both.Down.KEGG.tex}}
        \end{table}
    \end{frame}

    \begin{frame}[allowframebreaks]
        \frametitle{Finding in Comparing Recur vs. Non-recur in LUSC}

        \begin{block}{NTS}
            \begin{enumerate}
                \item Highly up-regulated in Recur patients.
                \item Neurotensin.
                \item Association with non-gastrointestinal cancers \cite{NTS1}.
                \item Modulate lung cancer cell plasticity and heterogeneity \cite{NTS2}.
            \end{enumerate}
        \end{block}

        \begin{block}{NTRK3}
            \begin{enumerate}
                \item Highly down-regulated in Recur patients.
                \item Activation of NTRK3 in LUSC \cite{NTRK3-1}.
                \item NTRK3 mutation has association with immunotherapy in LUAD \cite{NTRK3-2}.
            \end{enumerate}
        \end{block}

        \begin{block}{RECQL4}
            \begin{enumerate}
                \item Highly up-regulated in Non-recur patients.
            \end{enumerate}
        \end{block}

        \begin{block}{ASPA}
            \begin{enumerate}
                \item Highly down-regulated in Non-recur patients.
            \end{enumerate}
        \end{block}
    \end{frame}

    \subsubsection{Within Recur in LUSC}
    \begin{frame}
        \frametitle{DEG List for CIS within Recur in LUSC}

        \begin{table}
            \caption{Up-regulated DEG for CIS within Recur in LUSC}
            \begin{tabular}{lrrr}
\toprule
 gene &  log2FoldChange &   pvalue &     padj \\
\midrule
MFAP4 &        6.77e+00 & 2.70e-09 & 3.72e-07 \\
 TBX2 &        5.90e+00 & 1.19e-05 & 3.40e-04 \\
SFTPC &        5.47e+00 & 8.66e-08 & 6.57e-06 \\
\bottomrule
\end{tabular}

        \end{table}

        \begin{table}
            \caption{Down-regulated DEG for CIS within Recur in LUSC}
            \begin{tabular}{lrrr}
\toprule
   gene &  log2FoldChange &   pvalue &     padj \\
\midrule
 AKR1C2 &       -7.44e+00 & 4.70e-12 & 1.74e-09 \\
 AKR1C1 &       -7.09e+00 & 2.35e-35 & 2.74e-31 \\
CYP4F11 &       -6.70e+00 & 1.95e-14 & 1.75e-11 \\
\bottomrule
\end{tabular}

        \end{table}
    \end{frame}

    \begin{frame}
        \frametitle{DEG List for Primary within Recur in LUSC}

        \begin{table}
            \caption{Up-regulated DEG for Primary within Recur in LUSC}
            \begin{tabular}{lrrr}
\toprule
   gene &  log2FoldChange &   pvalue &     padj \\
\midrule
 AKR1C2 &        6.66e+00 & 4.34e-11 & 9.07e-09 \\
 AKR1C1 &        6.59e+00 & 2.62e-27 & 3.06e-23 \\
CYP4F11 &        6.25e+00 & 3.61e-11 & 7.67e-09 \\
\bottomrule
\end{tabular}

        \end{table}

        \begin{table}
            \caption{Down-regulated DEG for Primary within Recur in LUSC}
            \begin{tabular}{lrrr}
\toprule
   gene &  log2FoldChange &   pvalue &     padj \\
\midrule
  SFTPC &       -5.54e+00 & 1.56e-09 & 1.83e-07 \\
  CCBE1 &       -5.36e+00 & 9.73e-15 & 7.11e-12 \\
FAM107A &       -5.27e+00 & 3.64e-26 & 2.13e-22 \\
\bottomrule
\end{tabular}

        \end{table}
    \end{frame}

    \begin{frame}
        \frametitle{DEG Volcano Plots with Recur in LUSC}

        \begin{figure}
            $\begin{array}{ccc}
                \includegraphics[width=0.3 \linewidth]{figures/DEG/Volcano/STAR.SQC.Recur.Normal-CIS.volcano.pdf}
                &
                \includegraphics[width=0.3 \linewidth]{figures/DEG/Volcano/STAR.SQC.Recur.Normal-Primary.volcano.pdf}
                &
                \includegraphics[width=0.3 \linewidth]{figures/DEG/Volcano/STAR.SQC.Recur.CIS-Primary.volcano.pdf}
                \\
                \mbox{(a) Normal-CIS} & \mbox{(b) Normal-Primary} & \mbox{(c) CIS-Primary} \\
            \end{array}$
            \caption{DEG Volcano Plots with Recur samples in LUSC}
        \end{figure}
    \end{frame}

    \begin{frame}
        \frametitle{DEG Venn Diagram with Recur in LUSC}

        \begin{figure}
            $\begin{array}{ccc}
                \includegraphics[width=0.3 \linewidth]{figures/DEG/Pair-Venn/STAR.SQC.Recur.Up.venn.pdf}
                &
                \includegraphics[width=0.3 \linewidth]{figures/DEG/Pair-Venn/STAR.SQC.Recur.venn.pdf}
                &
                \includegraphics[width=0.3 \linewidth]{figures/DEG/Pair-Venn/STAR.SQC.Recur.Up.venn.pdf}
                \\
                \mbox{(a) Up-regulated} & \mbox{(b) Both} & \mbox{(c) Down-regulated} \\
            \end{array}$
            \caption{DEG Venn Diagram with Recur samples in LUSC}
        \end{figure}
    \end{frame}

    \begin{frame}
        \frametitle{Enrichment test with Normal vs. CIS for Recur}

        \begin{table}
            \caption{Up-regulated Pathways on Normal vs. CIS for Recur in LUSC}
            \resizebox{\linewidth}{!}
            {\begin{tabular}{llr}
\toprule
                                 Term name (24) &         Overlapping genes... &  Adjusted p-value \\
\midrule
                    Hypertrophic cardiomyopathy &     EDN1,TNNC1,ITGA1,...(15) &          5.17e-06 \\
                         Dilated cardiomyopathy & TNNC1,ITGA1,CACNA2D2,...(15) &          6.35e-06 \\
Arrhythmogenic right ventricular cardiomyopathy &    CACNG8,CACNB4,DES,...(12) &          9.65e-05 \\
\bottomrule
\end{tabular}
}
        \end{table}

        \begin{table}
            \caption{Down-regulated Pathways on Normal vs. CIS for Recur in LUSC}
            \resizebox{\linewidth}{!}
            {\begin{tabular}{lr}
\toprule
         Term name &  Adjusted p-value \\
\midrule
 Parkinson disease &          2.11e-05 \\
 Alzheimer disease &          2.11e-05 \\
Huntington disease &          2.11e-05 \\
\bottomrule
\end{tabular}
}
        \end{table}
    \end{frame}

    \begin{frame}
        \frametitle{Enrichment test with Normal vs. Primary for Recur}

        \begin{table}
            \caption{Up-regulated Pathways on Normal vs. Primary for Recur in LUSC}
            \resizebox{\linewidth}{!}
            {\begin{tabular}{llr}
\toprule
                Term name (8) &        Overlapping genes... &  Adjusted p-value \\
\midrule
 Glycolysis / Gluconeogenesis &      GPI,TPI1,PGAM1,...(13) &          3.06e-03 \\
Pathways of neurodegeneration &      RYR1,APP,COX7B,...(43) &          3.41e-03 \\
            Parkinson disease & COX7B,NDUFA10,GNAI1,...(27) &          3.74e-03 \\
\bottomrule
\end{tabular}
}
        \end{table}

        \begin{table}
            \caption{Down-regulated Pathways on Normal vs. Primary for Recur in LUSC}
            \resizebox{\linewidth}{!}
            {\begin{tabular}{llr}
\toprule
            Term name (20) &      Overlapping genes... &  Adjusted p-value \\
\midrule
   Cell adhesion molecules & ICAM2,NRXN3,PTPRM,...(25) &          1.28e-06 \\
Hematopoietic cell lineage &  CSF1R,CSF3,IL1R1,...(17) &          1.46e-04 \\
    Dilated cardiomyopathy & LAMA2,TNNC1,ADCY4,...(16) &          3.02e-04 \\
\bottomrule
\end{tabular}
}
        \end{table}
    \end{frame}

    \begin{frame}[allowframebreaks]
        \frametitle{Finding in Comparing within Recur in LUSC}

        \begin{block}{AKR1C1}
            \begin{enumerate}
                \item Down-regulated in CIS, but up-regulated in Primary.
            \end{enumerate}
        \end{block}

        \begin{block}{ADAM23}
            \begin{enumerate}
                \item Down-regulated in CIS, but up-regulated in Primary.
            \end{enumerate}
        \end{block}

        \begin{block}{FAM107A}
            \begin{enumerate}
                \item Up-regulated in CIS, but down-regulated in Primary.
            \end{enumerate}
        \end{block}
    \end{frame}

    \subsubsection{Within Non-recur in LUSC}
    \begin{frame}
        \frametitle{DEG List for CIS within Non-recur in LUSC}

        \begin{table}
            \caption{Up-regulated DEG for CIS within Non-recur in LUSC}
            \begin{tabular}{lrrr}
\toprule
    gene &  log2FoldChange &   pvalue &     padj \\
\midrule
   SFTPC &        3.89e+00 & 1.33e-08 & 1.10e-06 \\
   MUCL3 &        3.64e+00 & 4.99e-18 & 6.71e-15 \\
HLA-DRB1 &        3.40e+00 & 3.00e-05 & 6.50e-04 \\
\bottomrule
\end{tabular}

        \end{table}

        \begin{table}
            \caption{Down-regulated DEG for CIS within Non-recur in LUSC}
            \begin{tabular}{lrrr}
\toprule
  gene &  log2FoldChange &   pvalue &     padj \\
\midrule
AKR1C1 &       -6.10e+00 & 1.95e-19 & 4.73e-16 \\
AKR1C2 &       -5.81e+00 & 4.57e-17 & 5.03e-14 \\
   NTS &       -5.19e+00 & 1.60e-13 & 8.61e-11 \\
\bottomrule
\end{tabular}

        \end{table}
    \end{frame}

    \begin{frame}
        \frametitle{DEG List for Primary within Non-recur in LUSC}

        \begin{table}
            \caption{Up-regulated DEG for Primary within Non-recur in LUSC}
            \begin{tabular}{lrrr}
\toprule
  gene &  log2FoldChange &   pvalue &     padj \\
\midrule
AKR1C1 &        6.10e+00 & 9.04e-23 & 6.57e-20 \\
AKR1C2 &        5.91e+00 & 3.13e-19 & 9.92e-17 \\
   NTS &        5.78e+00 & 2.01e-14 & 2.28e-12 \\
\bottomrule
\end{tabular}

        \end{table}

        \begin{table}
            \caption{Down-regulated DEG for Primary within Non-recur in LUSC}
            \begin{tabular}{lrrr}
\toprule
   gene &  log2FoldChange &   pvalue &     padj \\
\midrule
  SFTPC &       -5.89e+00 & 5.10e-20 & 1.86e-17 \\
 LRRC36 &       -4.57e+00 & 2.29e-33 & 1.42e-29 \\
FAM107A &       -4.51e+00 & 2.49e-30 & 7.01e-27 \\
\bottomrule
\end{tabular}

        \end{table}
    \end{frame}

    \begin{frame}
        \frametitle{DEG Volcano Plots with Non-recur in LUSC}

        \begin{figure}
            $\begin{array}{ccc}
                \includegraphics[width=0.3 \linewidth]{figures/DEG/Volcano/STAR.SQC.Nonrecur.Normal-CIS.volcano.pdf}
                &
                \includegraphics[width=0.3 \linewidth]{figures/DEG/Volcano/STAR.SQC.Nonrecur.Normal-Primary.volcano.pdf}
                &
                \includegraphics[width=0.3 \linewidth]{figures/DEG/Volcano/STAR.SQC.Nonrecur.CIS-Primary.volcano.pdf}
                \\
                \mbox{(a) Normal-CIS} & \mbox{(b) Normal-Primary} & \mbox{(c) CIS-Primary} \\
            \end{array}$
            \caption{DEG Volcano Plots with Non-recur samples in LUSC}
        \end{figure}
    \end{frame}

    \begin{frame}
        \frametitle{DEG Venn Diagram with Non-recur in LUSC}

        \begin{figure}
            $\begin{array}{ccc}
                \includegraphics[width=0.3 \linewidth]{figures/DEG/Pair-Venn/STAR.SQC.Nonrecur.Up.venn.pdf}
                &
                \includegraphics[width=0.3 \linewidth]{figures/DEG/Pair-Venn/STAR.SQC.Nonrecur.venn.pdf}
                &
                \includegraphics[width=0.3 \linewidth]{figures/DEG/Pair-Venn/STAR.SQC.Nonrecur.Up.venn.pdf}
                \\
                \mbox{(a) Up-regulated} & \mbox{(b) Both}& \mbox{(c) Down-regulated} \\
            \end{array}$
            \caption{DEG Venn Diagram with Non-recur in LUSC}
        \end{figure}
    \end{frame}

    \begin{frame}
        \frametitle{Enrichment test with Normal vs. CIS for Non-recur}

        \begin{table}
            \caption{Up-regulated Pathways on Normal vs. CIS for Non-recur in LUSC}
            \resizebox{\linewidth}{!}
            {\begin{tabular}{llr}
\toprule
                  Term name &   Overlapping genes... &  Adjusted p-value \\
\midrule
                    Malaria & CSF3,HGF,ITGB2,...(14) &          6.53e-08 \\
 Hematopoietic cell lineage & CSF1R,CSF3,MME,...(18) &          2.01e-07 \\
Hypertrophic cardiomyopathy & EDN1,ACE,LAMA2,...(16) &          1.53e-06 \\
\bottomrule
\end{tabular}
}
        \end{table}

        \begin{table}
            \caption{Down-regulated Pathways on Normal vs. CIS for Non-recur in LUSC}
            \resizebox{\linewidth}{!}
            {\begin{tabular}{lr}
\toprule
                                   Term name &  Adjusted p-value \\
\midrule
Metabolism of xenobiotics by cytochrome P450 &          9.67e-05 \\
                             Drug metabolism &          1.18e-04 \\
                                  Cell cycle &          1.89e-04 \\
\bottomrule
\end{tabular}
}
        \end{table}
    \end{frame}

    \begin{frame}
        \frametitle{Enrichment test with Normal vs. Primary for Non-recur}

        \begin{table}
            \caption{Up-regulated Pathways on Normal vs. Primary for Non-recur in LUSC}
            \resizebox{\linewidth}{!}
            {\begin{tabular}{llr}
\toprule
          Term name (30) &       Overlapping genes... &  Adjusted p-value \\
\midrule
              Cell cycle &  HDAC1,PKMYT1,ORC4,...(30) &          4.62e-06 \\
Homologous recombination &   BLM,PALB2,RAD54B,...(14) &          1.52e-04 \\
         DNA replication & FEN1,RNASEH2A,RFC4,...(13) &          1.52e-04 \\
\bottomrule
\end{tabular}
}
        \end{table}

        \begin{table}
            \caption{Down-regulated Pathways on Normal vs. Primary for Non-recur in LUSC}
            \resizebox{\linewidth}{!}
            {\begin{tabular}{llr}
\toprule
             Term name (58) &        Overlapping genes... &  Adjusted p-value \\
\midrule
 Hematopoietic cell lineage &    CSF1R,CSF3,CSF3R,...(29) &          1.24e-10 \\
Hypertrophic cardiomyopathy & LAMA2,ITGB3,CACNA1D,...(26) &          1.44e-09 \\
                    Malaria &       IL10,CSF3,CR1,...(18) &          2.11e-08 \\
\bottomrule
\end{tabular}
}
        \end{table}
    \end{frame}

    \begin{frame}[allowframebreaks]
        \frametitle{Finding in Comparing within Non-recur in LUSC}

        \begin{block}{AKR1C1 \& AKR1C2}
            \begin{enumerate}
                \item Down-regulated in CIS, but up-regulated in Primary.
            \end{enumerate}
        \end{block}

        \begin{block}{CYP4F11}
            \begin{enumerate}
                \item Down-regulated in CIS, but up-regulated in Primary.
            \end{enumerate}
        \end{block}

        \begin{block}{LRRC36}
            \begin{enumerate}
                \item Up-regulated in CIS, but down-regulated in Primary.
            \end{enumerate}
        \end{block}
    \end{frame}

    \subsubsection{Within Non-recur in LUAD}
    \begin{frame}
        \frametitle{LUAD Data Composition}

        \begin{table}
            \caption{Number of WTS LUAD samples}
            \begin{tabular}{l|lr}
Recurrence? & Stage & Number of samples \\ \hline
\multirow{5}{*}{Recurrence (n=4)} & Normal & 1 \\
 & AAH & 0 \\
 & AIS & 2 \\
 & MIA & 0 \\
 & Primary & 1 \\ \hline
\multirow{5}{*}{Non-recurrence (n=26)} & Normal & 11 \\
 & AAH & 1 \\
 & AIS & 7 \\
 & MIA & 0 \\
 & Primary & 7
\end{tabular}
        \end{table}
    \end{frame}

    \begin{frame}
        \frametitle{DEG List for AIS within Non-recur in LUAD}

        \begin{table}
            \caption{Up-regulated DEG for AIS within Non-recur in LUAD}
            \begin{tabular}{lrrr}
\toprule
    gene &  log2FoldChange &   pvalue &     padj \\
\midrule
    MUC4 &        4.83e+00 & 2.55e-04 & 1.68e-02 \\
   SIPA1 &        4.77e+00 & 4.87e-05 & 6.37e-03 \\
C11orf45 &        4.68e+00 & 2.86e-04 & 1.85e-02 \\
\bottomrule
\end{tabular}

        \end{table}

        \begin{table}
            \caption{Down-regulated DEG for AIS within Non-recur in LUAD}
            \begin{tabular}{lrrr}
\toprule
  gene &  log2FoldChange &   pvalue &     padj \\
\midrule
 ABCA4 &       -5.02e+00 & 2.44e-10 & 5.29e-07 \\
UNC13C &       -4.08e+00 & 6.49e-06 & 1.88e-03 \\
SLC7A5 &       -3.93e+00 & 1.40e-06 & 6.76e-04 \\
\bottomrule
\end{tabular}

        \end{table}
    \end{frame}

    \begin{frame}
        \frametitle{DEG List for Primary within Non-recur in LUAD}

        \begin{table}
            \caption{Up-regulated DEG for Primary within Non-recur in LUAD}
            \begin{tabular}{lrrr}
\toprule
 gene &  log2FoldChange &   pvalue &     padj \\
\midrule
ABCA4 &        5.22e+00 & 1.67e-11 & 3.32e-08 \\
HMGA2 &        5.03e+00 & 4.39e-07 & 9.62e-05 \\
KIF12 &        4.54e+00 & 2.62e-06 & 3.91e-04 \\
\bottomrule
\end{tabular}

        \end{table}

        \begin{table}
            \caption{Down-regulated DEG for Primary within Non-recur in LUAD}
            \begin{tabular}{lrrr}
\toprule
   gene &  log2FoldChange &   pvalue &     padj \\
\midrule
 SLC6A4 &       -5.92e+00 & 3.83e-08 & 1.47e-05 \\
TINAGL1 &       -5.27e+00 & 9.47e-06 & 9.57e-04 \\
 SFTPA1 &       -4.91e+00 & 2.69e-04 & 1.13e-02 \\
\bottomrule
\end{tabular}

        \end{table}
    \end{frame}

    \begin{frame}
        \frametitle{DEG Volcano Plots with Non-recur in LUAD}

        \begin{figure}
            $\begin{array}{ccc}
                \includegraphics[width=0.3 \linewidth]{figures/DEG/Volcano/STAR.ADC.Nonrecur.Normal-AIS.volcano.pdf}
                &
                \includegraphics[width=0.3 \linewidth]{figures/DEG/Volcano/STAR.ADC.Nonrecur.Normal-Primary.volcano.pdf}
                &
                \includegraphics[width=0.3 \linewidth]{figures/DEG/Volcano/STAR.ADC.Nonrecur.AIS-Primary.volcano.pdf}
                \\
                \mbox{(a) Normal-AIS} & \mbox{(b) Normal-Primary} & \mbox{(c) AIS-Primary} \\
            \end{array}$
            \caption{DEG Volcano Plots with Non-recur samples in LUAD}
        \end{figure}
    \end{frame}

    \begin{frame}
        \frametitle{DEG Venn Diagram with Non-recur in LUAD}

        \begin{figure}
            $\begin{array}{ccc}
                \includegraphics[width=0.3 \linewidth]{figures/DEG/Pair-Venn/STAR.ADC.Nonrecur.Up.venn.pdf}
                &
                \includegraphics[width=0.3 \linewidth]{figures/DEG/Pair-Venn/STAR.ADC.Nonrecur.venn.pdf}
                &
                \includegraphics[width=0.3 \linewidth]{figures/DEG/Pair-Venn/STAR.ADC.Nonrecur.Down.venn.pdf}
                \\
                \mbox{(a) Up-regulated} & \mbox{(b) Both} & \mbox{(c) Down-regulated} \\
            \end{array}$
            \caption{DEG Venn Diagram with Non-recur in LUAD}
        \end{figure}
    \end{frame}

    \begin{frame}
        \frametitle{Enrichment test with Normal vs. AIS in LUAD}

        \begin{table}
            \caption{Up-regulated Pathways on Normal vs. AIS for Non-recur in LUAD}
            \resizebox{\linewidth}{!}
            {\begin{tabular}{llr}
\toprule
                Term name & Overlapping genes... &  Adjusted p-value \\
\midrule
Calcium signaling pathway & NTRK2,RYR2,CHRM1,... &          3.90e-02 \\
\bottomrule
\end{tabular}
}
        \end{table}

         \begin{table}
            \caption{Down-regulated Pathways on Normal vs. AIS for Non-recur in LUAD}
            \resizebox{\linewidth}{!}
            {\input{tables/RSEM/STAR.ADC.Nonrecur.Normal-AIS.Down.KEGG.tex}}
        \end{table}
    \end{frame}

    \begin{frame}
        \frametitle{Enrichment test with Normal vs. Primary in LUAD}

        \begin{table}
            \caption{Up-regulated Pathways on Normal vs. Primary for Non-recur in LUAD}
            \resizebox{\linewidth}{!}
            {\input{tables/RSEM/STAR.ADC.Nonrecur.Normal-Primary.Up.KEGG.tex}}
        \end{table}

        \begin{table}
            \caption{Down-regulated Pathways on Normal vs. Primary for Non-recur in LUAD}
            \resizebox{\linewidth}{!}
            {\begin{tabular}{lr}
\toprule
                         Term name &  Adjusted p-value \\
\midrule
          ECM-receptor interaction &          2.05e-03 \\
Vascular smooth muscle contraction &          4.98e-03 \\
         Calcium signaling pathway &          7.82e-03 \\
\bottomrule
\end{tabular}
}
        \end{table}
    \end{frame}

    \begin{frame}[allowframebreaks]
        \frametitle{Finding in Comparing within Non-recur in LUAD}

        \begin{block}{KCNQ3}
            \begin{enumerate}
                \item Down-regulated in AIS, but up-regulated in Primary.
            \end{enumerate}
        \end{block}

        \begin{block}{BLACAT1}
            \begin{enumerate}
                \item Down-regulated in AIS, but up-regulated in Primary.
            \end{enumerate}
        \end{block}
    \end{frame}

    \begin{frame}
        \frametitle{Findings in DEG Analysis}
    \end{frame}

    \subsection{Bulk Cell Deconvolution}
    \begin{frame}
        \frametitle{Single-cell data as Reference}

        \begin{figure}
            \includegraphics[width=0.8 \linewidth]{figures/CIBERSORTx/reference.jpg}
            \caption{Comprehensive dissection and clustering of 208,506 single cells from LUAD patients \protect\cite{singlecell1}}
        \end{figure}
    \end{frame}

    \begin{frame}
        \frametitle{BisqueRNA?}

        \begin{figure}
            \includegraphics[width=0.6 \linewidth]{figures/Workflow/Bisque.jpg}
            \caption{Workflow for BisqueRNA \protect\cite{Bisque1}}
        \end{figure}
    \end{frame}

    \begin{frame}
        \frametitle{Cluster Plot in LUSC}

        \begin{figure}
            \includegraphics[height=0.6 \textheight]{figures/BisqueRNA/clustermap/STAR.SQC.cluster.pdf}
            \caption{Cluster Plot in LUSC}
        \end{figure}
    \end{frame}

    \begin{frame}[allowframebreaks]
        \frametitle{Violin Plots in LUSC}

        \begin{figure}
            $\begin{array}{ccc}
                \includegraphics[width=0.3 \linewidth]{figures/BisqueRNA/violin/STAR.SQC.violin/ActivatedDCs.pdf}
                &
                \includegraphics[width=0.3 \linewidth]{figures/BisqueRNA/violin/STAR.SQC.violin/AT2.pdf}
                &
                \includegraphics[width=0.3 \linewidth]{figures/BisqueRNA/violin/STAR.SQC.violin/CD1c+DCs.pdf}
                \\
                \mbox{(a) Activated DCs} & \mbox{(b) Alveolar type II} & \mbox{(c) Langerhans cells} \\
            \end{array}$
            \caption{Violin Plots in LUSC}
        \end{figure}

        \begin{figure}
            $\begin{array}{ccc}
                 \includegraphics[width=0.3 \linewidth]{figures/BisqueRNA/violin/STAR.SQC.violin/Ciliated.pdf}
                &
                \includegraphics[width=0.3 \linewidth]{figures/BisqueRNA/violin/STAR.SQC.violin/ExhaustedTfh.pdf}
                &
                \includegraphics[width=0.3 \linewidth]{figures/BisqueRNA/violin/STAR.SQC.violin/mo-Mac.pdf}
                \\
                \mbox{(d) Ciliated cells} & \mbox{\tiny (e) Exhausted T follicular helper} & \mbox{(f) Mo \& Mac} \\
            \end{array}$
            \caption{Violin Plots in LUSC}
        \end{figure}

        \begin{figure}
            $\begin{array}{ccc}
                \includegraphics[width=0.3 \linewidth]{figures/BisqueRNA/violin/STAR.SQC.violin/pDCs.pdf}
                &
                \includegraphics[width=0.3 \linewidth]{figures/BisqueRNA/violin/STAR.SQC.violin/PleuralMac.pdf}
                &
                \includegraphics[width=0.3 \linewidth]{figures/BisqueRNA/violin/STAR.SQC.violin/Treg.pdf}
                \\
                \mbox{(g) Plasmacytoid DCs} & \mbox{(h) Pleural Mac} & \mbox{(i) Regulatory T cells} \\
            \end{array}$
            \caption{Violin Plots in LUSC}
        \end{figure}

        \begin{figure}
            $\begin{array}{ccc}
                \includegraphics[width=0.3 \linewidth]{figures/BisqueRNA/violin/STAR.SQC.violin/tS1.pdf}
                &
                &
                \\
                \mbox{(j) Transcriptional states 1} &  & \\
            \end{array}$
            \caption{Violin Plots in LUSC}
        \end{figure}
    \end{frame}

    \begin{frame}[allowframebreaks]
        \frametitle{Findings in Bulk Cell Deconvolution with LUSC}

        \begin{block}{Activated DCs}
            content...
        \end{block}

        \begin{block}{Alveolar type II}
            content...
        \end{block}

        \begin{block}{CD1c+ DCs (Langerhans cells)}
            content...
        \end{block}

        \begin{block}{Cilited cells}
            content...
        \end{block}

        \begin{block}{Exhausted T follicular help}
            content...
        \end{block}

        \begin{block}{Monocyte \& Macrophage}
            content...
        \end{block}

        \begin{block}{Transcriptional states 1}
            content...
        \end{block}
    \end{frame}

    \begin{frame}
        \frametitle{Cluster Plot in LUAD}

        \begin{figure}
            \includegraphics[height=0.6 \textheight]{figures/BisqueRNA/clustermap/STAR.ADC.cluster.pdf}
            \caption{Cluster Plot in LUAD}
        \end{figure}
    \end{frame}

    \begin{frame}[allowframebreaks]
        \frametitle{Violin Plots in LUAD}

        \begin{figure}
            $\begin{array}{ccc}
                \includegraphics[width=0.3 \linewidth]{figures/BisqueRNA/violin/STAR.ADC.violin/Ciliated.pdf}
                &
                \includegraphics[width=0.3 \linewidth]{figures/BisqueRNA/violin/STAR.ADC.violin/Club.pdf}
                &
                \includegraphics[width=0.3 \linewidth]{figures/BisqueRNA/violin/STAR.ADC.violin/MAST.pdf}
                \\
                \mbox{(a) Ciliated cells} & \mbox{(b) Club Cell} & \mbox{(c) Mast cell} \\
            \end{array}$
            \caption{Violin Plots in LUAD}
        \end{figure}

        \begin{figure}
            $\begin{array}{ccc}
                \includegraphics[width=0.3 \linewidth]{figures/BisqueRNA/violin/STAR.ADC.violin/NK.pdf}
                &
                \includegraphics[width=0.3 \linewidth]{figures/BisqueRNA/violin/STAR.ADC.violin/Smoothmusclecells.pdf}
                &
                \includegraphics[width=0.3 \linewidth]{figures/BisqueRNA/violin/STAR.ADC.violin/Tip-likeECs.pdf}
                \\
                \mbox{(d) NK cells} & \mbox{(e) Smooth muscle cells} & \mbox{(f) Tip-like ECs} \\
            \end{array}$
            \caption{Violin Plots in LUAD}
        \end{figure}
    \end{frame}

    \begin{frame}[allowframebreaks]
        \frametitle{Findings in Bulk Cell Deconvolution with LUAD}

        \begin{block}{Cilited cells}
            content...
        \end{block}

        \begin{block}{Club cells}
            content...
        \end{block}

        \begin{block}{Mast cells}
            content...
        \end{block}

        \begin{block}{Natural Killer cells}
            content...
        \end{block}

        \begin{block}{Smooth muscle cells}
            content...
        \end{block}

        \begin{block}{Tip-like ECs}
            content...
        \end{block}
    \end{frame}

    \begin{frame}
        \frametitle{Findings in Bulk Cell Deconvolution}
    \end{frame}

    \subsection{Discovery of Gene Fusion}
    \begin{frame}
        \frametitle{Arriba?}

        \begin{figure}
            \includegraphics[width=0.6 \linewidth]{figures/Workflow/Arriba.png}
            \caption{Benchmark of Arriba versus alternative methods \protect\cite{Arriba1}}
        \end{figure}
    \end{frame}

    \begin{frame}
        \frametitle{Findings in Gene Fusion Discovery}
    \end{frame}

    \section{Discussion}

    \section{References}
    \begin{frame}[allowframebreaks]
        \frametitle{References}
        \bibliographystyle{apacite}
        \bibliography{reference}
    \end{frame}
\end{document}